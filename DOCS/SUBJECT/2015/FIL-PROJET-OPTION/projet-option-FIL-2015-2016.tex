\documentclass[a4paper,11pt]{article}

% paquets perso
%\usepackage{clavierFonte}
%\usepackage{guillemets}
\usepackage[utf8]{inputenc}
\usepackage[T1]{fontenc}
\usepackage[french]{babel}
\newcommand{\fge}{\fg\ }


% paquet / note EMN
\usepackage{sujetProjet}

\usepackage{pdfpages}

\newcommand{\aegis}{\texttt{Aegis}}
\newcommand{\api}{\texttt{API}}
\newcommand{\cxf}{\texttt{Apache CXF}}
\newcommand{\ccxf}{\texttt{CXF}}
\newcommand{\corba}{\texttt{CORBA}}
\newcommand{\emn}{EMN}
\newcommand{\eclipse}{\texttt{Eclipse}}
\newcommand{\gsi}{GSI}
\newcommand{\discovery}{DisCoVery }
\newcommand{\http}{\texttt{HTTP}}
\newcommand{\java}{\texttt{Java}}
\newcommand{\jaxb}{\texttt{JAXB}}
\newcommand{\jaxws}{\texttt{JAX-WS}}
\newcommand{\jaxrs}{\texttt{JAX-RS}}
\newcommand{\jbi}{\texttt{JBI}}
\newcommand{\jee}{\texttt{JEE}}
\newcommand{\javaJEE}{\java{}/\jee{}}
\newcommand{\jms}{\texttt{JMS}}
\newcommand{\json}{\texttt{JSON}}
\newcommand{\restful}{\texttt{RESTful}}
\newcommand{\soap}{\texttt{SOAP}}
\newcommand{\uv}{UV}
\newcommand{\xml}{\texttt{XML}}




\initTitre{UV \og Sensibilisation à la recherche \fge \\ BtrPlaceS  \\ Simulateur pour l'étude d'algorithmes de placements de machines virtuelles}
\initAbrevTitre{Projet BtrPlaceS}

%\initSousTitre{}

\initAuteurs{}

%\initDate{22}{8}{2006}

\initSousEntite{FIL}
\initCodeSousEntite{FIL}

%\initDestinataires{}

%\initDiffusion{}

%\initCodeNote{UV/}

%\initVersionNote{1.0}

%\initHistorique{}

\pasIndentation

\begin{document}

\begin{couverture}
  \afficherDate
  \afficherTitre

{\footnotesize
  \begin{sujet}

\afficherTuteur{%
\href{mailto:adrien.lebre@inria.fr}{Adrien Lebre}
}


\afficherProjet{%
 BtrPlaceS

 \begin{quote}
   L'étude d'algorithmes de placements de machines virtuelles dans les
   infrastructures dites de ``cloud computing'' est un axe de recherche
   particulièrement actif. Avec plus de 150 articles de recherche depuis 2008,
   il est difficile pour les chercheurs de déterminer les avantages des
   nouvelles propositions par rapport à l'état de l'art. C'est dans ce contexte
   que l'équipe ASCOLA de l'école des Mines de Nantes a proposé la solution
   VMPlaceS: un simulateur générique dédiée à l'évaluation et la comparaison
   d'algorithmes de placements de machines virtuelles. Disponible en
   open-source, VMplaceS est livré avec trois exemples permettant aux chercheurs
   de comprendre comment mettre en oeuvre et évaluer de nouvelles stratégies.
   Néanmoins, ces trois solutions reposent sur un algorithme qui n'est plus maintenu.
   Le travail proposé consiste à remplacer cet algorithme de résolution par celui
   utilisé dans le système BtrPlace.

   A plus gros grain, ce projet s'insère dans l'initiative
\href{http://beyondtheclouds.github.io}{Discovery} pilotée par l'école.
 \end{quote}
}

\afficherSiteWeb{http://beyondtheclouds.github.io}

\afficherIntitule{%
  Remplacement de l'algorithme de résolution par défaut dans le système VMPlaceS
}

\end{sujet}
}
\end{couverture}


%\includepdf[pages=-, offset=2cm 0cm]{KeyCopyKey-projet.pdf}

\begin{note}


\paragraph*{Domaines}

\begin{itemize}
 \item Intégration logicielle
 \item Cloud Computing (IaaS)
 \item Java
\end{itemize}

\paragraph*{Compétences requises}

\begin{itemize}
 \item Fondamentaux Cloud Computing
 \item Programmation objets/composants
 \item JAVA 
\end{itemize}

\paragraph*{Compétences à acquérir}

\begin{itemize}
 \item Prise en main de l'EDI \texttt{IntelliJ} et des outils de collaboration Github
 \item Sensibilistation au un simulateur à évènements discrets
 \item Contribution à un projet \og~open source~\fg
\end{itemize}

\paragraph*{Contexte\\}
   Le cloud computing peut-être vu comme une solution permettant aux entreprises
   d'externaliser leurs ressources informatiques dans des infrastructures
   distantes et ce au travers d'environnements virtualisés (communément appelés
   machines virtuelles). Le fournisseur du service cloud est en charge de
   l'éxécution de ces dernières au dessus des ressources physiques qu'il
   possède, son objectif étant généralement de maximiser son bénéfice (c'est à
   dire de déterminer le meilleur compromis entre les gains que lui rapporte la
   location de ces machines virtuelles par rapport aux coûts liés à leurs
   exécutions sur les machines physiques). Ce problème d'optimisation est traité
   au travers d'algorithmes de placements de machines virtuelles. Un tel
   algorithme sert à déterminer et à maintenir un affectation des machines
   virtuelles sur les machines physiques afin de garantir d'une part que les
   ressources escomptées par les machines virtuelles puissent être affectées et
   d'autres part de manière à optimiser un critère propre au fournisseur
   (critère économique comme préalablement cité, énergétique ou encore
   équilibrage de charge pour des applications spécifiques). De part
   l'importance de ces algorithmes pour les fournisseurs, un nombre significatif
   de solutions a été proposé et il est malheureusement aujourd'hui très
   compliqué d'arriver à identifier les avantages et inconvénients de chacune
   d'entre elles.

\paragraph*{VMPlaceS: un simulateur dédiée à l'étude d'algorithmes de placements\\}
   Depuis 2014, l'équipe ASCOLA développe un simulateur générique ayant pour
   principal objectif le développement et l'évaluation de nouvelles techniques
   de placements de machines virtuelles. Intitulé VMPlaceS, cet outil a été
   officiellement présenté en 2015 lors de la conférence Europar. Sa pertinence
   a été démontré au travers la comparaison de trois stratégies référencées à
   plusieurs reprises dans la littérature. Néanmoins l'algorithme d'optimisation
   utilisé dans cette preuve de concept repose sur le système Entropy qui n'est plus maintenu. Il est donc primordiale afin de favoriser
   l'adoption de l'outil VMPlaceS par la communauté, de
   remplacer cet algorithme par une solution plus récente qui pourra être
   comparée avec des nouvelles stratégies. 
 
 

\paragraph*{Objectif\\}
L'objectif du travail proposé dans ce projet vise à remplacer l'algorithme
actuel par celui utilisé dans le système BtrPlace. BtrPlace propose une version
``revisitée'' de l'algorithme Entropy en y intégrant des nouvelles abstractions
offrant une plus grande flexibilité aux administrateurs dans la gestion des
machines virtuelles et également une plus grande efficacité dans le processus de résolution.
Le projet sera articulé autour de deux volets: 
\begin{itemize}
\item La lecture et la compréhension de deux articles scientifiques (référencés ci-après) ; 
\item Le remplacement de l'algorithme caduque au sein du code de VMPlaceS par l'algorithme utilisé dans BtrPlace.
\end{itemize}

\paragraph*{Méthodologie du projet\\}
Ce projet sera réalisé suivant une méthode agile. Il devra impliquer une
contribution au projet \og~open source~\fg VMPlaceS.


\paragraph*{Bibliographie:\\}
%
\texttt{[1]}Fabien Hermenier, Julia Lawall et Gilles Muller. 
Btrplace: A Flexible Consolidation Manager for Highly Available Applications, IEEE Transactions of Dependable and Secure Computing (TDSC), vol 10, no. 5, pp. 273--286, Sept.-Oct. 2013.\\
\href{https://sites.google.com/site/hermenierfabien/btrplace-tdsc2013.pdf}{https://sites.google.com/site/hermenierfabien/btrplace-tdsc2013.pdf}

\texttt{[2]} Adrien Lebre, Jonathan Pastor, et Mario Südholt. VMPlaceS: A Generic Tool to Investigate and Compare VM Placement Algorithms. Europar 2015, Août 2015, Vienne, Autriche.\\
\href{http://people.rennes.inria.fr/Adrien.Lebre/PREPRINT/Europar-VMPlaceS.pdf}{http://people.rennes.inria.fr/Adrien.Lebre/PREPRINT/Europar-VMPlaceS.pdf}
\end{note}



\end{document}
