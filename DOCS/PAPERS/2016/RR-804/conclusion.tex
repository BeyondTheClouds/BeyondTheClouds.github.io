%%%
\section{Conclusion\label{sec:conclusion}}

Distributing the management of Clouds is a solution to favor the adoption of the distributed cloud model. In this paper, we presented our view of how
such distribution can be achieved by presenting the premises of the LUC Operating System.
%We highlighted that it has  however a design cost
%and it  should be  developed over  mature and efficient  solutions:
We chose to develop it by leveraging the OpenStack solution. This choice
presents two advantages. It minimizes the development efforts and maximizes the
chance of being reused by a large community. As a proof-of-concept we presented
a revised version of the Nova service that uses a NoSQL backend. We discussed
few experiments validating the correct behavior and showed promising performance
over 8 clusters.
 
Our ongoing activities focus on two aspects. First, we expect
to finalize the same modifications on the Glance image service soon and start to
investigate also whether such a DB replacement can be achieved for Neutron. We
highlight that we chose to concentrate our effort on Glance as it is a key
element to operate an OpenStack IaaS platform. Indeed, while Neutron is becoming
more and more important, the historical network mechanisms integrated in Nova
are still available and intensively used. The second activity studies how it can
be possible to restrain the visibility of some objects manipulated by the
different controllers that have been deployed throughout the LUC infrastructure:
%Indeed some Finally, having a wan-wide infrastructure can be source of
%networking overheads:
our POC manipulates objects that might be used by any instance of a service, no
matter where it is deployed.
%On the  other hand, some  objects may
%benefit from a  restrained visibility:
If a user has build an OpenStack project (tenant) that is based on few sites,
appart from data-replication, there is no need for storing objects related to
this project on external sites. Restraining the storage of such objects
according to visibility rules would save network bandwidth and reduce
overheads.% in addition to enabling users to settle policies for applications
%such as privacy and efficient data-replication.

Although delivering an efficient distributed version of OpenStack is a
challenging task, we believe that addressing it is the key to go beyond
classical brokering/federated approaches and to promote a new generation of
cloud computing more sustainable and efficient. We are in touch with large
groups such as Orange Labs and are currently discussing with the OpenStack
foundation to propose the Rome/REDIS Library as an alternative to the
SQLAlchemy/MySQL couple. Most of the materials presented in this article such as
our prototype are available on the Discovery initiative website.

%Indeed,
%revising  OpenStack, in  order to  make  it natively  cooperative, would  enable
%Internet  Service Providers  and other  institutions  in charge  of operating  a
%network backbone  to build  an extreme-scale LUC  infrastructure with  a limited
%additional cost.

%% The  interest of important actors such as  Orange Labs that has
%% officially announced its  support to the initiative is an  excellent sign of the
%% importance of our action.       
