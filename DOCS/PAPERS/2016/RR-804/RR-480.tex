\documentclass[a4paper,twoside]{article}

\usepackage{RR}
\usepackage{hyperref}
%\usepackage[frenchb]{babel}
%\usepackage[T1]{fontenc} % avec T1 comme option  d'encodage c'est ben mieux, surtout pour taper du français.
\usepackage[utf8]{inputenc}
\usepackage[table]{xcolor}
\usepackage{color}
\usepackage{graphicx}
\usepackage{amsmath, amsthm}
\usepackage{stmaryrd}
\usepackage{lscape}

\usepackage{url} \urlstyle{sf}%
\usepackage{graphicx}%
\usepackage{subfigure}
\usepackage{listings}%

\lstset{%
  basicstyle=\scriptsize,%
  numbers=left,
  %columns=fullflexible,%
  language=XML,%
  %frame=shadowbox,
    frame=lbtr,%
  frameround=tttt,%
  tabsize=2,
  breaklines=true
}%
\usepackage{tikz}
\usetikzlibrary{decorations.pathreplacing,positioning}
\usepackage{array}
\usepackage{xspace}

\newcommand{\ie}[0]{{\em i.e.},\xspace}
\newcommand{\vs}[0]{{\em vs.}\xspace}
\newcommand{\eg}[0]{{\em e.g.},\xspace}
\newcommand{\etal}[0]{{\em et al.}\xspace}
\newcommand{\wrt}[0]{{\em w.r.t.}\xspace}
\newcommand{\aka}[0]{{\em a.k.a.}\xspace}

\sloppy

%%% FORMAT DOCUMENT

\def\textfraction{0}
\def\floatpagefraction{1}
\def\topnumber{3}
\def\bottomnumber{3}
\def\totalnumber{3}
\def\topfraction{1}
\def\bottomfraction{1}

%%.
\usepackage{bold-extra,graphicx,latexsym,mathrsfs,subfigure,xspace}

\usepackage{color}
\usepackage{array}
\usepackage{longtable}
\usepackage{calc}
\usepackage{multirow}
\usepackage{hhline}
\usepackage{ifthen}

\usepackage{hyperref}

\newcolumntype{M}[1]{>{\raggedleft}m{#1}}
\newcommand{\discovery}{DISCOVERY\xspace}

%% CT %%
\newcommand{\ftodo}[2][\relax]  % \TODO[editor]{text} 
  {\ensuremath{{}^{\mbox{\tiny\bf #1}}}~\textbf{#2}}

\begin{document}

\RRNo{480}
\RRdate{February 2016}

\RRprojet{DISCOVERY IPL}

% \author{Adrien L?bre\inst{1,3} \and Jonathan Pastor\inst{1,3} \and Marin Bertier \inst{2,3} \and Fr?d?ric Desprez\inst{3} \and Jonathan Rouzaud-Cornabas\inst{3} \and C?dric Tedeschi \inst{3,4}\and Paolo Anedda\inst{5} \and Gianluigi Zanetti\inst{5} 
% \and Ramon Nou\inst{6} \and Toni Cortes\inst{6} \and Etienne Riviere\inst{7} \and Thomas Ropars\inst{8}}
% \institute{LINA / Mines Nantes, France 
% \and IRISA / INSA de Rennes, France
% \and INRIA, France
% \and IRISA / Universit? de Rennes 1, France
% \and Center for Advanced Studies, Research and Development in Sardinia (CRS4), Italy
% \and Barcelona Supercomputing Center (BSC), Spain
% \and Universit? de Neuch?tel (UniNe), Switzerland
% \and Ecole Polytechnique F?d?rale de Lausanne (EPFL), Switzerland} 

%
% Title and Authors
%
\RRauthor{
A. Lebre\thanks[Inria]{Inria, France, Email: \url{FirstName.LastName@inria.fr}}\thanks[EMN]{Mines Nantes/LINA (UMR 6241), France.}%\inst{1} 
\and 
J. Pastor\thanksref{Inria}\thanksref{EMN} 
\and 
Frederic Desprez\thanksref{Inria}  
}

\authorhead{A. Lebre et al.}

\RRtitle{Révisiter les mécanismes internes du système OpenStack en vue d'opérer des infrastructures de type nuage massivement distribuées}
\RRetitle{
A Ring to Rule Them All - Revising OpenStack Internals to Operate Massively Distributed Clouds}%\thanks{This report ....}}
\titlehead{The DISCOVERY Initiative}

%\RRnote {XXXX}

\RRkeyword{Fog, Edge Computing, 
Peer To Peer,
Self-*,
Sustainability,
Efficiency,
OpenStack, 
Future Internet.
}

\RRmotcle{Calcul utilitaire basé sur la localité, systèmes pair-à-pair, self-*, durabilité,OpenStack 
Internet du futur}

%
% Abstract
%
\RRabstract{

  {\small 

The deployment of micro/nano data-centers in network point of presence offers an opportunity to deliver a more sustainable and efficient infrastructure for Cloud Computing. Among the different challenges we need to address to favor the adoption of such a model, the development of a system in charge of turning such a complex and diverse network of resources into a collection of abstracted computing facilities that are convenient to administrate and use is critical.

In this report, we introduce the premises of such a system. The novelty of our work is that instead of developing a system from scratch, we revised the OpenStack solution in order to operate such an infrastructure in a distributed manner leveraging P2P mechanisms. More precisely, we describe how we revised the Nova service by leveraging a distributed key/value store instead of the centralized SQL backend. We present experiments that validated the correct behavior of our prototype, while having promising performance using several clusters composed of servers of the Grid'5000 testbed. We believe that such a strategy is promising and paves the way to a first large-scale and WAN-wide IaaS manager.

}
}

\RRresume{

  {\small 

La tendance actuelle pour supporter la demande croissante d'informatique
utilitaire consiste à construire des centres de données de plus en plus grands,
dans un nombre limité de lieux stratégiques. Cette approche permet sans aucun
doute de satisfaire la demande actuelle tout en conservant une approche
centralisée de la gestion de ces ressources, mais elle reste loin de pouvoir
fournir des infrastructures répondant aux contraintes actuelles et futures en
termes d’efficacité, de juridiction ou encore de durabilité.  L’objectif de
l'initiative DISCOVERY\footnote{\url{http://beyondtheclouds.github.io}} est de
concevoir le \emph{LUC OS},  un système de gestion distribuée des ressources qui permettra de
tirer parti de n’importe quel n\oe ud réseau constituant la dorsale d’Internet
afin de fournir une nouvelle génération d’informatique utilitaire, plus apte à
prendre en compte la dispersion géographique des utilisateurs et leur demande
toujours croissante. 

Après avoir rappelé les objectifs de l'initiative DISCOVERY et expliqué
pourquoi les approches type fédération ne sont pas adaptées pour opérer une
infrastructure d'informatique utilitaire intégrée au réseau, nous présentons les
prémisses de notre système.  Nous expliquerons notamment pourquoi et comment
nous avons choisi de démarrer des travaux visant à revisiter la conception de
la solution Openstack. De notre point de vue, choisir d'appuyer nos travaux sur
cette solution est une stratégie judicieuse à la vue de la complexité des
systèmes de gestion des plateformes IaaS et de la vélocité des solutions
open-source. 
}
}

\URRennes
\makeRR

%%%

\subsection{Discovery Initiative}

\begin{itemize}

	\item Users manipulates virtual environment, which is the "gravity center of the IaaS".

	\item Virtual environment contains virtual machines.

\end{itemize}

\section{Designing a massively distributed Cloud}
\label{sec:design}


\subsection{Toolkit for IaaS}

% \begin{itemize}

% 	\item A toolkit is a building block that can be used for the construction of
% 	systems (generic definition of a software toolkit).

% 	\item The objective of a toolkit is to provide "state of the art" solutions
% 	to known problems. It enables the focus on "Top level" works.

% 	\item It provides a set of components, which once assembled constitute an 
% 	operational system.

% 	\item Recent studies of "state of art IaaS systems" (OpenStack, Cloudstack,
% 	OpenNebula, ...) showed that they were constructed over same concepts. It 
% 	enables the design of IaaS toolkit.

% 	\item The massively distributed IaaS toolkit will provides "state of the 
% 	arts" mechanism to solve both scalability and locality points.

% 	\item The toolkit will have to integrate well on existing systems: we 
% 	propose to leverage OpenStack project.

% \end{itemize}

A software toolkit is a set of software building blocks that includes state of 
the art mechanisms for known problems. A toolkit comes with an API (Application
Programming Interface) which is the specifications that ones must follow to
correctly use provided mechanisms. The goal of a toolkit is to enable developers
to focus on the creation of higher level mechanisms, thus speeding up the 
development time.

An IaaS toolkit should be delivered with a set of default high level mechanisms 
whose assembly results in an basic operational IaaS system. In the case where 
one of the default constituting mechanisms would not be sufficient, it should be
redeveloped by leveraging the toolkit's low level mechanisms.

Recent studies have showed that state of the art IaaS manager \cite{peng:2009}
were constructed over the same concepts. Furthermore a reference architecture 
for IaaS manager has been described in \cite{moreno2012iaas} enabling the design
of an IaaS toolkit. Besides the reference architecture, an IaaS toolkit should
provides some mechanisms that enables to solve the scalability problem.

To maximize the chance of being reused by a large community, an IaaS toolkit 
should enable an easy integration with one or several existing IaaS cloud
managers. In our case we have chosen to leverage the OpenStack project: as a 
result the mechanisms developped with the toolkit will integrate well with 
existing clouds that are based on OpenStack.



\subsection{Massively distributed cloud}

% \begin{itemize}

% 	\item A massively distributed cloud targets management of thousand of hosts 
% 	around a wide territory.

% 	\item This scale order is currently reached by file sharing systems like 
% 	bittorrent.

% 	\item At this scale, failure becomes the norm.

% 	\item Recent works propose to leverage on peer to peer overlay.

% 	\item Some peer to peer overlays can take advantage of locality. It enables
% 	to build systems that can take into account network bandwidth and latency.

% 	\item We propose to leverage on locality based peer to peer mechanisms to 
% 	reach an high scalability IaaS.

% \end{itemize}

Cloud providers concentrate the production of computing resources in 
data-centers that contains tens of thousand of servers, enabling IaaS mechanisms
to take advantage of fast network with extremely low latency. However this ever 
increasing data-centers size has become a problem, as many data-centers require 
dedicated electrical and cooling infrastructure. As an alternative to 
concentrating the production of computing resources, we propose to study a model
where this production is deconcentrated.

Leveraging the concept of micro data-centers proposed by \cite{greenberg:2008},
we suggest to build a cloud operating system that will run in a distributed
manner over a set a small data-centers geographically spread. This cloud 
operating system will have to reach high scalability criteria: managing 
thousands of servers used by hundreds of users. Popular peer to peer file
sharing systems already work at this scale order: bittorrent clients enable 
hundreds of thousands of users to share millions of file spread over the 
internet. If we disregard trackers, this protocol is totally decentralized with 
no single point of failure (SPOF). That is why we propose to learn from peer to 
peer file sharing experience, in order to build massively distributed clouds.

As at this scale failure becomes the norm rather the exception, it is vital to
take into account fault tolerance in the early stages of design, by leveraging a
peer to peer overlay network. As we think that working in a massively
distributed context require to deal with network parameters like 
latency and bandwidth usage, collaboration between the constituting nodes of the
system should be organized in a "network aware" manner. For instance, the 
Vivaldi algorithm \cite{dabek:2004:vivaldi} provides a dynamic coordinate system
that can be used to introduce locality properties inside a distributed system, 
thus building low latency collaborations.

We assume that the architecture of massively distributed clouds should be 
organized arround the same principles that rule communautary file sharing 
systems. That is why we propose in section \ref{sec:architecture} an 
architecture that will be build on top of peer to peer principles and 
articulated arround a locality based overlay network. To meet the high 
scalability criteria, some of IaaS mechanisms should be revisited to improve 
reactivity of inter-servers collaboration by leveraging locality properties.





\section{Revisiting OpenStack: towards a massively distributed IaaS manager}


\subsection{OpenStack: a toolkit for building clouds}

OpenStack is a very popular project that aim at building an opensource IaaS
manager. Many of the biggest actors of Cloud computing (Red hat, IBM, Rackspace,
VMware, Cisco, ...) are contributing to this project, thus providing new 
features at a rapide pace. OpenStack enables the construction of public or 
private clouds.

A typical cloud deployed with OpenStack is composed of several services (nova, 
swift, Quantum, Glance, ...) following the "shared nothing architecure" 
principle, meaning that each service is independent and thus shares no state 
with other services.

In this way, inter-services collaboration is performed by exchanging message 
through an AMQP (Advanced Messsage Queuing Protocol) based bus: each service 
has its own queue, and it collaborates with others by sending messages to their 
queues. This offers the advantage of easily plugging additional components : 
when a default service becomes no more suitable with system's needs, it can be 
naively replaced by another custom service, as long as the newer service 
consummes messages destinated to the older one, and in turn produces messages 
expected by collaborators.

As OpenStack is becoming a standard and since it provides mechansims that are
suitable to build clouds in data-centers context, it can be considerated as a
cloudkit of choice for the LUC-OS. 












\subsection{Revisiting components that are suitable for a LUC-OS (Related work)}
\label{sub:sec:revisiting_openstack}


As developing a complete IaaS manager is an Herculean work, we think that 
revisiting existing mechanims is a good way to reach our goal, even if it means 
that some of these mechanisms may be completely replaced.

For the sake of clarity, mechanisms that we plan to revisit will be studied 
following the list of services depicted in section \ref{sub:sec:list_services} :
for each service, the suitable mechanisms that will be revisited are indicated 
and proposed modification are detailled:

\begin{description}

	\item [Compute manager] : Nova is the OpenStack service that is 
	responsible for coordinating and controlling the work of other services.
	As it is the angular stone of OpenStack, we plan to directly plug the 
	LUC-OS on this service. Nova service contains a sub-service that is
	responsible for scheduling virtual machines: nova-scheduler, which performs
	static scheduling. This means that the scheduler will decide on which
	server the virtual machine will run once and for all. In section 
	\ref{sub:sec:integration_dvms} we discuss the replacement of 
	nova-scheduler by DVMS.


	\item [Administrative manager] : KeyStone and Horizon are two services
	that are respectively responsibles for managing identity/authentication
	and providing a web user interface to users that want to interact with
	OpenStack. As the LUC-OS will store every piece of information in a DHT: we
	propose to make KeyStone and Horizon pick data from this DHT. Furthermore 
	the LUC-OS and OpenStack does not exactly share the same semantics : for
	example the LUC-OS manipulates virtual environments which does not
	exactly exist in OpenStack. Revisiting Horizon service through minor
	modifications would enable the integration of this concept in OpenStack.

	\item [Storage manager] : Storage in OpenStack can be divided in two
	service: Swift manages objects (files) while Glance manages virtual machines
	disk images. Glance's images can be stored in Swift or in any other storage
	service. Recent solutions like VMTorrent \cite{reich:2012} have been
	developed to limit network overhead, for example during simultaneous 
	loading of images at VMs startup. For a first prototype we propose to 
	leverage Glance and Swift, and if it becomes no more suitable for a
	massive infrastucture, we propose to integrate VMTorrent in Glance and 
	to use it.

	\item [Network manager] : Neutron is the service that manages networking
	in OpenStack: it is used to create virtual networks (VLANs) that
	interconnect virtual machines. Neutron supports plugins : it is possible
	to use a virtual multilayer switch like Open vSwitch \cite{pfaff:2009}.
	In the case this plugin does not meet our needs, we propose to 
	leverage a software defined network (SDN) solution like Mininet.
	\cite{lantz:2010} or Vin.

\end{description}

\section{Experimental validation on Grid'5000}
\label{sec:eval}

The validation of our prototype has been performed thanks to the Grid'5000
testbed \cite{grid5000}. Grid'5000 is a large-scale and versatile experimental
testbed for experiment driven research in Computer Science, which enables
researchers to get an access to a large amount of computing resources
($\sim$ 1000 nodes spread over 10 sites). This delivery of computing resources
takes the form of bare-metal machines, on which fully customised software stacks
can deployed, thus giving a very fine control of the experimental conditions.

Various tools have been developed to provide an ease of use, such as monitoring
information about networking and power consumption or programming libraries to
fine tune each aspect composing an experiment. With this in mind, we developed
our prototype using the Execo framework \cite{imbert:hal-00861886} which helped
us to deploy and configure each node composing our chosen software stack (Ubuntu
14.04, a modified version of OpenStack "devstack", and the RIAK key/value store).

Our validation scenario has consisted in the creation of 30 VMs
through 10 different Nova controllers deployed on one cluster located
in Nantes. This experiment enabled us to confirm that OpenStack's
services were working correctly with the key/value store.
% that the relational backend
%used by the several components can be satisfactorily replaced by this
%new distributed key/value system.
Although further experiments are
required to test the scalability as well as the effect
of geographical distances on the reliability and efficiency, we are
confident about our approach as several distributed key/value stores
are already used WANWide.

% \AL[JP]{Finalize the text of this section, please be consistent the
%   announcement at the end of the introduction}
% Several test-cases with 10 nodes to evaluate
% the efficiency of the new framework:\begin{inparaenum}[1\upshape)]
% \item 1 site, 1 controller, 9 compute nodes
% \item 1 site, 10 controllesr, 0 compute node
% \item 2 sites, 1 controller/by site, 4 compute nodes/by site
% \item and 2 sites, 5 controllers/by site, 0 compute nodes/by site
% \end{inparaenum}



%test infrastructure is a tedious task if one has to do it
%manually. In order to simplify tests, we create a tool based on \texttt{Execo}
%\cite{imbert:hal-00861886} that: find a time slot when the resources
%are available on the several sites.
% deploy Ubuntu 14.04 on all the nodes, install RIAK DB and setup
% our modified version of devstack (with automatic node configuration), and
% finally start the distributed OpenStack infrastructure. The duration of the
% deployment is around 1 hour, depending on the cluster hardware and the
% total number of nodes. Finally, the tool provides some options that simplifies
% the management of the different test-cases, namely the number of sites,
% controller by site and compute nodes by site.

%\begin{figure}[h!]
%    \centering
%    \includegraphics[width=7cm]{figures/6_sites.png}
%    \caption{Test-cases on the Grid'5000 platform involving 6
%    geographically-spreaded sites each hosting 2 controllers and 10 computes
%    nodes.}
%\end{figure}


%\subsubsection{Test-cases}
%\paragraph{Monosite}
%\paragraph{Multisite: 1 controller per site}
%\paragraph{Multisite: 10 controllers per site}


\section{On-going Work}
\label{sec:ongoing_work}

While the Nova revision we presented in the previous
sections is a promising proof-of-concept toward widely distributed OpenStack
infrastructures, there are remaining challenges that need to be tackle.

%First, it is critical to study whether similar changes can be achieved on other components.
%Second, the notion of locality does not exist in our current implementation.
%That is, any request can be served by any controller available, leading to the creation of a VM anywhere in the infrastructure.

In  this section,  we present  two on-going  actions that  aim at  deepening the
relevance of our approach. First,  we show that existing seggregation mechanisms
provided by OpenStack are not satisfactory  when it comes to reducing inter-site
communications.
%  in  order to  improve  networking  efficiency of  a  multi-site OpenStack  deployment.
In  response,  two actions  could be  made:  on one  hand
introducing  networking  locality  in  the   shared  databases  and  the  shared
messaging, on the other hand distributing remaining services of OpenStack.

%While the achievement of such
%changes are under heavy development, we believe that our approach is promising
%enough to favor the adoption of the distributed cloud model supervised by a
%single system.

Second, we discuss the preliminary study we  made on the Glance image service to
investigate whether it is possible to apply similar modifications to the ones we
performed on Nova. Such a validation is  critical as Glance is a key element for
operating a production-ready IaaS infrastructure.

These two actions clearly demonstrate that our approach is promising
enough to favor the adoption of the distributed cloud model supervised by a
single system.

\subsection{Locality Challenges / $\mu$cro DCs Segregation}

%% Even if this article has demonstrated that the relational database used by Nova
%% could be replaced by a non relational key/value store, a more advanced
%% validation of this change is required and the question of which metrics to use
%% remains. On one hand, a larger deployment involving more geographical sites
%% would be more demonstrative, on the other hand the changes we introduced have
%% been have been motivated by more than overall scalabity, but also fault
%% tolerance and reduction of response times when processing API requests.

%% \subsection{Distributing the remaining services of OpenStack}

%% As introduced in section \ref{leveraging-openstack}, the other services
%% composing OpenStack are historically leveraging a relational database to store
%% their inner-states. As done with Nova, this relational database may be replaced
%% by a key/value data-store. Among the remaining services, the next candidate is
%% the image service Glance: as its images are already stored in fully distributed
%% cloud storage software (SWIFT), the next step to reach a fully distributed
%% functionning with Glance is to apply the same strategy that we did with Nova. On
%% the other  hand, the situation may be different with some other services:
%% Neutron works  with drivers that may be intented to work in a distributed way.
%% In such  situation alternatives have to be found.

%% \subsection{Locality aware objects in OpenStack}

%% Having a wan-wide infrastructure can be source of networking overheads: some
%% objects manipulated by OpenStack are subject to be manipulated by any service of
%% the deployed controllers, and by extension should be visible to any of the
%% controllers. On the other hand, some objects may benefit from a restrained
%% visibility: if a user has build an OpenStack project (tenant) that is based on
%% few sites, appart from data-replication there is no need for storing objects
%% related to this project on external sites. Restraining the storage of such
%% objects according to visibility rules would enable to save network bandwidth and
%% to settle policies for applications such as privacy and efficient data-
%% replication.

Deploying  a massively  multi-site  Cloud Computing  infrastructure operated  by
OpenStack   is  challenging   as  communication   between  nodes   of  different
geographical clusters can be subject to  an important network latency, which can
be a source of disturbances for OpenStack. Experimental results presented in the
Table~\ref{tab:orgtable2}  of  Section~\ref{sec:multisite_exps}  clearly  showed
that an  OpenStack distributed  on top  of our  Rome+REDIS solution  can already
operate over an  ISP network with a high inter-site  latency (50~ms). While this
result is  positive and can indicate  that such a configuration  is appropriated
for operating  a distributed  CC infrastructure  involving tens  of geographical
site,  it   is  important   to  understand  the   nature  of   network  traffic.
Table~\ref{tab:experiments-host-aggregates-network}  shows   the  total  traffic
\textit{vs.} the  traffic between  the remote  sites using  a 4  sites OpenStack
leveraging  our Rome+REDIS  proposal and  with the  host-aggregate feature.  The
first  line clearly  shows that  even with  the host-aggregate  feature enabled,
there is a dramatic amount of communications (87.7\%) made between nodes located
in distinct geographical sites.

\begin{table}[htb]
\vspace{-0.5cm}
\caption{\label{tab:experiments-host-aggregates-network}
  Quantity of data exchanged over network (in MBytes)}
\centering
\begin{tabular}{lrrr}
 & Total  & Inter-site &  Proportion \\
\hline
%without host-aggregates & 4484 & 4171 & 93.0\% \\
%with host-aggregates & 5326 & 4672 & 87.7\% \\
4 clusters  & 5326 & 4672 & 87.7\% \\
\end{tabular}
\end{table}

%% The  first  approach that  came  to  mind to  reduce  the  amount of  inter-site
%% communication is  to seggregate nodes  composing the infrastructure  by grouping
%% nodes of a same site or from close  sites. We tested such a strategy by grouping
%% each  node of  a same  site in  a same  host-aggregates/availability-zone, which
%% enables to provide a finer control on the scheduling of a VM. The second line of
%% Table~\ref{tab:experiments-host-aggregates-network} shows  that despite  a minor
%% reduction of the proportion of inter-sites  communications, it remains at a very
%% high level (87.7\%).

A quarter of these inter-site communications are caused by the isolation of Nova
from other OpenStack services (\ie Keystone and Glance) which were deployed on a
dedicated master node in our experiments. Indeed, operations like serving VM
images were naturally a source of artificial inter-site communications. This
situation clearly advocates in favor of massively distributing the remaining
services, as we did with Nova. Finally, as instances of OpenStack services
collaborate via a shared messaging bus and via a shared database, unless these
two elements will be able to avoid the broadcasting of information by taking
advantage of network locality, the level of inter-site communication will remain
large. We are investigating two directions. First, we are studying whether the
use of a P2P bus such as ZeroMQ\cite{zeromq:2013} can reduce such a network overhead and second
whether the service catalog of Keystone can become locality-aware in order to ``hide'' redundant services that are located remotely.


\subsection{Revising Glance: The OpenStack Image Manager}
Similarly to the Nova component (see Section \ref{subsec:mysql-to-redis}), only
the inner states of Glance are stored in a MySQL DB, the VM images are already
stored in a fully distributed way (leveraging either SWIFT or Ceph/Rados
solution \cite{weil2006ceph}). Therefore, our preliminary study aimed at
determining whether it was possible or not to reuse the ROME library to switch
between the SQL and NoSQL backends. As depicted by Figure \ref{fig:glance_dbs},
the Glance code from the software engineering point of view is rather close to
the Nova one. As a consequence, replacing the MySQL DB by a KVS system did not
lead to specific issues. We underline that the replacement of MySQL with REDIS
was even more straightforward than for Nova as Glance enables the configuration
of specific API for accessing persistent data (\texttt{data\_api} in the Glance
configuration file). We are currently validating that each request is correctly
handled by Rome. Preliminary performance experiments are planned for the
beginning of 2016.

%API to use for accessing data. Default value points to sqlalchemy
%# package, it is also possible to use: glance.db.registry.api
%# data_api = glance.db.sqlalchemy.api

\begin{figure}[htbp]
%\vspace*{-0.3cm} 
        \centering
        \includegraphics[width=8.5cm]{figures/rome_glance.png}
%\vspace*{-0.8cm}
        \caption{Glance - Software Architecture and DB dependencies.}
        \label{fig:glance_dbs}
%\vspace*{-.3cm}
\end{figure}


\section{Future work}
\label{sec:future_work}

\subsection{Deeper validation with larger testbed}

Even if this article has demonstrated that the relational database used by Nova
could be replaced by a non relational key/value store, a more advanced
validation of this change is required and the question of which metrics to use
remains. On one hand, a larger deployment involving more geographical sites
would be more demonstrative, on the other hand the changes we introduced have
been have been motivated by more than overall scalabity, but also fault
tolerance and reduction of response times when processing API requests.

\subsection{Distributing the remaining services of OpenStack}

As introduced in section \ref{leveraging-openstack}, the other services
composing OpenStack are historically leveraging a relational database to store
their inner-states. As done with Nova, this relational database may be replaced
by a key/value data-store. Among the remaining services, the next candidate is
the image service Glance: as its images are already stored in fully distributed
cloud storage software (SWIFT), the next step to reach a fully distributed
functionning with Glance is to apply the same strategy that we did with Nova. On
the other  hand, the situation may be different with some other services:
Neutron works  with drivers that may be intented to work in a distributed way.
In such  situation alternatives have to be found.

\subsection{Locality aware objects in OpenStack}

Having a wan-wide infrastructure can be source of networking overheads: some
objects manipulated by OpenStack are subject to be manipulated by any service of
the deployed controllers, and by extension should be visible to any of the
controllers. On the other hand, some objects may benefit from a restrained
visibility: if a user has build an OpenStack project (tenant) that is based on
few sites, appart from data-replication there is no need for storing objects
related to this project on external sites. Restraining the storage of such
objects according to visibility rules would enable to save network bandwidth and
to settle policies for applications such as privacy and efficient data-
replication.

 

%%%
\section{Conclusion\label{sec:conclusion}}

Distributing the management of Clouds is a solution to favor the adoption of the distributed cloud model. In this paper, we presented our view of how
such distribution can be achieved by presenting the premises of the LUC Operating System.
%We highlighted that it has  however a design cost
%and it  should be  developed over  mature and efficient  solutions:
We chose to develop it by leveraging the OpenStack solution. This choice
presents two advantages. It minimizes the development efforts and maximizes the
chance of being reused by a large community. As a proof-of-concept we presented
a revised version of the Nova service that uses a NoSQL backend. We discussed
few experiments validating the correct behavior and showed promising performance
over 8 clusters.
 
Our ongoing activities focus on two aspects. First, we expect
to finalize the same modifications on the Glance image service soon and start to
investigate also whether such a DB replacement can be achieved for Neutron. We
highlight that we chose to concentrate our effort on Glance as it is a key
element to operate an OpenStack IaaS platform. Indeed, while Neutron is becoming
more and more important, the historical network mechanisms integrated in Nova
are still available and intensively used. The second activity studies how it can
be possible to restrain the visibility of some objects manipulated by the
different controllers that have been deployed throughout the LUC infrastructure:
%Indeed some Finally, having a wan-wide infrastructure can be source of
%networking overheads:
our POC manipulates objects that might be used by any instance of a service, no
matter where it is deployed.
%On the  other hand, some  objects may
%benefit from a  restrained visibility:
If a user has build an OpenStack project (tenant) that is based on few sites,
appart from data-replication, there is no need for storing objects related to
this project on external sites. Restraining the storage of such objects
according to visibility rules would save network bandwidth and reduce
overheads.% in addition to enabling users to settle policies for applications
%such as privacy and efficient data-replication.

Although delivering an efficient distributed version of OpenStack is a
challenging task, we believe that addressing it is the key to go beyond
classical brokering/federated approaches and to promote a new generation of
cloud computing more sustainable and efficient. We are in touch with large
groups such as Orange Labs and are currently discussing with the OpenStack
foundation to propose the Rome/REDIS Library as an alternative to the
SQLAlchemy/MySQL couple. Most of the materials presented in this article such as
our prototype are available on the Discovery initiative website.

%Indeed,
%revising  OpenStack, in  order to  make  it natively  cooperative, would  enable
%Internet  Service Providers  and other  institutions  in charge  of operating  a
%network backbone  to build  an extreme-scale LUC  infrastructure with  a limited
%additional cost.

%% The  interest of important actors such as  Orange Labs that has
%% officially announced its  support to the initiative is an  excellent sign of the
%% importance of our action.       


%
% Section: Acknowledgment
%
\section*{Acknowledgments}

Since July 2015, the Discovery initiative is mainly supported through the Inria Project Labs program and the I/O labs, a joint lab between Inria and Orange Labs.
Further information at \href{http://www.inria.fr/en/research/research-fields/inria-project-labs}{http://www.inria.fr/en/research/research-fields/inria-project-labs}
%
% Bibliography
%



\bibliographystyle{abbrv}
\bibliography{main}
\end{document}
