%% Try to take into account Toni remarks
%%Contrary to the current trend that promotes large offshore-centralized data
%%centers as the Utility Computing (UC) infrastructure of choice, we claim the
%%only way to achieve sustainable and highly efficient UC services is to target a
%%new infrastructure that better match the geographical dispersal of users as
%%well as the ever increasing demand for computing resources.  Although it
%%involves radical changes in the way resources are managed, we propose to
%%leverage any facilities available through the Internet: Combinning network and
%%UC resources is the only solution to deliver a new generation of UC
%%infrastructures, highly efficient and sustainable.  
%%
%
%
%To overcome the lack of sustainable solutions that are able to accommodate the
%ever increasing demand for computing resources, a disruptive change in Utility
%Computing (UC) infrastructures is required. Locality has to be considered as a
%primary concern. Instead of the current trend consisting of building larger and
%larger data centers in few strategic locations, we propose to leverage any
%facilities available through the Internet in order to deliver widely
%distributed UC platforms that can better match the geographical dispersal of
%users as well as the unending demand. Although it involves radical changes in
%the way resources are managed, leveraging computing resources close to the
%end-users is the only solution to deliver a new generation of UC platforms,
%highly efficient and sustainable. 
%%
%Critical to the emergence of such locality-based UC (LUC) infrastructures is
%the availability of appropriate operating mechanisms. In this paper, we
%advocate for the implementation a unified system driving the use of resources
%at an unprecedented scale by turning a complex and diverse infrastructure into
%a collection of abstracted computing facilities that is both easy to operate
%and reliable. The ultimate objective of our vision is to make possible to
%host/operate a large part of the Internet by its internal structure itself.
%Similarly to the interconnection of academics and private networks and the use
%of the TCP/IP standard that resulted in the Internet, the deployment and the
%use of the \discovery proposal, an agent based system leveraging autonomous, decentralized
%and self-organizing techniques, might lead to the \discovery infrastructure:  A
%scalable and nearly infinite set of resources delivered by any computing
%facilities forming the Internet, starting from the larger hubs operated by
%ISPs, government and academic institutions to any idle resources that may be
%provided by end-user. 


%From the pad
%To accomodate the ever-increasing demand for computing resources, while
%taking  into account both energy and economical issues, the current trend consists in building larger and larger data centers in a few strategic locations. Although such an approach enables to cope with the ever-increasing demand of Utility Computing (UC) resources while continuing to operate them through centralized software system, we believe that they will not be able to support the overall demand in a near future.
%
%Utility computing as it has been defined in the past is far from being available in a sustainable and efficient way. A disruptive change in UC infrastructures is required: (i) Locality has to be considered as a primary concern (ii) A different way of resources management is mandatory, leveraging computing resources close to the end-users. To this aim, we propose to leverage any facilities available through the Internet in order to deliver widely distributed UC platforms that can better match the geographical dispersal of users as well as the unending demand.
%
%Critical to the emergence of such locality-based UC (LUC) infrastructures is the availability of appropriate operating mechanisms. In this paper, we advocate for the implementation a unified system driving the use of resources at an unprecedented scale by turning a complex and diverse infrastructure into a collection of abstracted computing facilities that is both easy to operate and reliable. The ultimate objective of our vision is to make possible to host/operate a large part of the Internet by its internal structure itself. Similarly to the interconnection of academics and private networks and the use of the TCP/IP standard that resulted in the Internet, the deployment and the use of the DISCOVERY proposal, an agent based system leveraging autonomous, decentralized and self-organizing techniques, might lead to the DISCOVERY infrastructure: A scalable and nearly infinite set of resources delivered by any computing facilities forming the Internet, starting from the larger hubs operated by ISPs, government and academic institutions to any idle resources that may be provided by end-user.
%

To accommodate the ever-increasing demand for Utility Computing (UC) resources, while taking
into account both energy and economical issues, the current trend consists in
building larger and larger data centers in a few strategic locations. Although
such an approach enables to cope with the actual demand 
while continuing to operate UC resources through centralized
software system, it is far from delivering sustainable and efficient UC infrastructures. 

We claim that a disruptive change in UC infrastructures is required: UC
resources should be managed differently, considering locality as a
primary concern. We propose to leverage any facilities available through the
Internet in order to deliver widely distributed UC platforms that can better
match the geographical dispersal of users as well as the unending demand.
Critical to the emergence of such locality-based UC (LUC) platforms is
the availability of appropriate operating mechanisms. In this paper, we
advocate the implementation of a unified system driving the use of
resources at an unprecedented scale by turning a complex and diverse
infrastructure into a collection of abstracted computing facilities that is
both easy to operate and reliable. 
%
By deploying and using such a \emph{LUC Operating System} on backbones,
our ultimate vision is to make possible to host/operate a large part of the
Internet by its internal structure itself: A scalable and nearly infinite set
of resources delivered by any computing facilities forming the Internet,
starting from the larger hubs operated by ISPs, government and academic
institutions to any idle resources that may be provided by end-users. 
%
Unlike previous researches on distributed operating systems, we propose to
consider virtual machines (VMs) instead of processes as the basic element.  System
virtualization offers several capabilities that increase the flexibility of
resources management, allowing to investigate novel decentralized schemes% that
% will enable to achieve the LUC infrastructure we target
.

