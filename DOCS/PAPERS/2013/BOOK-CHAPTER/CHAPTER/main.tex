%%%%%%%%%%%%%%%%%%%% author.tex %%%%%%%%%%%%%%%%%%%%%%%%%%%%%%%%%%%
%
% sample root file for your "contribution" to a contributed volume
%
% Use this file as a template for your own input.
%
% SPRINGER  RECOMMENDED %%%%%%%%%%%%%%%%%%%%%%%%%%%%%%%%%%%%%%%%%%%%%%%%%%%
\documentclass[graybox]{svmult}

% choose options for [] as required from the list
% in the Reference Guide

\usepackage{mathptmx}       % selects Times Roman as basic font
\usepackage{helvet}         % selects Helvetica as sans-serif font
\usepackage{courier}        % selects Courier as typewriter font
\usepackage{type1cm}        % activate if the above 3 fonts are
                            % not available on your system
%
\usepackage{makeidx}         % allows index generation
\usepackage{graphicx}        % standard LaTeX graphics tool
                             % when including figure files
\usepackage{multicol}        % used for the two-column index
\usepackage[bottom]{footmisc}% places footnotes at page bottom

% see the list of further useful packages
% in the Reference Guide

\makeindex             % used for the subject index
                       % please use the style svind.ist with
                       % your makeindex program
%%%%%%%%%%%%%%%%%%%%%%%%%%%%%%%%%%%%%%%%%%%%%%%%%%%%%%%%%%%%%%%%%%%%%%%%%%%%%%%%%%%%%%%%%

% ADDED BY OURSELVES %%%%%%%%%%%%%%%%%%%%%%%%%%%%%%%%%%%%%%%%%%%%%%%%%%%
%% Package WE added
\usepackage[utf8]{inputenc}
\usepackage{hyperref}
\usepackage{xspace}
%% Definition WE made
\newcommand{\discovery}{\mbox{DISCOVERY}\xspace}
\newcommand{\ie}{\textit{i.e.}\xspace}
%% CT %%
\newcommand{\ftodo}[2][\relax]  % \TODO[editor]{text} 
  {\ensuremath{{}^{\mbox{\tiny\bf #1}}}~\textbf{#2}}


\begin{document}

\title*{Beyond The Clouds, How Should Next Generation Utility Computing Infrastructures Be Designed?}
\titlerunning{Beyond The Clouds, The \discovery Initiative}
% Use \titlerunning{Short Title} for an abbreviated version of
% your contribution title if the original one is too long
\author{Marin Bertier, Frédéric Desprez, Gilles Fedak, Adrien Lebre, Anne-Cécile
  Orgerie, Jonathan Pastor, Flavien Quesnel, Jonathan Rouzaud-Cornabas, Cédric Tedeschi}
\authorrunning{Lebre et al.} 
% Use \authorrunning{} for an abbreviated version of
% your contribution title if the original one is too long
%\institute{Name of First Author \at Name, Address of Institute, \email{name@email.address}
%\and Name of Second Author \at Name, Address of Institute \email{name@email.address}}
\institute{INRIA, France, \email{firstname.lastname@inria.fr}}
%
% Use the package "url.sty" to avoid
% problems with special characters
% used in your e-mail or web address
%
\maketitle

\begin{abstract}
The use of computing resources provided by cloud companies has become very common. Cloud providers leverage on big datacenters containing tens of thousands of servers. Designing efficient software that can work at this scale is very complex: fault tolerance, network overhead and synchronization can have a significant cost.
\\
A possible solution to minimize these costs is to consider locality properties: servers collaborate first with close servers from the same geographical site. As a result, collaboration between servers can take into account infrastructure parameters such as bandwidth and response time and can be organized in a totally decentralized manner.
\\
This paper shows the case of a large scale virtual machine scheduling algorithm, DVMS, that have been adapted to take advantage of locality properties by using a vivaldi based network overlay. Results of experiments conducted on the grid5000 tesbed, comparing the chord based versus the vivaldi based algorithm, will be presented in the last section.

\keywords{Cloud computing, locality, peer to peer, network overlay, vivaldi, chord, DVMS, virtual machine scheduling}
\end{abstract}


%\abstract{Each chapter should be preceded by an abstract (10--15 lines long) that summarizes the content. The abstract will appear \textit{online} at \url{www.SpringerLink.com} and be available with unrestricted access. This allows unregistered users to read the abstract as a teaser for the complete chapter. As a general rule the abstracts will not appear in the printed version of your book unless it is the style of your particular book or that of the series to which your book belongs.\newline\indent
%Please use the 'starred' version of the new Springer \texttt{abstract} command for typesetting the text of the online abstracts (cf. source file of this chapter template \texttt{abstract}) and include them with the source files of your manuscript. Use the plain \texttt{abstract} command if the abstract is also to appear in the printed version of the book.}

\section{Context and Motivations}
\label{sec:intro}
The success of Cloud Computing has driven the advent of Utility Computing.
However, Cloud Computing is a victim of its own success: In order to answer the
escalating demand for computing resources, Cloud Computing providers must
build data centers~(DCs) of ever-increasing size.

Besides facing the well-known
issues of large-scale platforms management, large-scale DCs have to deal with
energy considerations that limit the number of physical resources that one
location can host.

Instead of investigating alternative solutions that can tackle the aforementioned 
concerns, the current trend consists in deploying larger and larger DCs in few
strategic locations presenting energy advantages.  For example, Western North
Carolina, USA, is an attractive area due to its abundant capacity of coal
and nuclear power following the departure of the textile and furniture
industry~\cite{greenpeace:2013}.  More recently, several proposals suggested building next
generation DCs close to the polar circle  in order to leverage free cooling
techniques, considering that cooling is accounting for a big part of the
electricity consumption~\cite{greenberg:sigcomm09}. 


\subsection{Inherent Limitations of Large-scale DCs}

Although building large scale DCs  enables to cope with the actual demand,
% while continuing to operate UC resources through centralized software system 
it is far from delivering sustainable and efficient UC infrastructures.  In addition
to requiring the construction and the deployment of a complete network
infrastructure to reach each DC, it exacerbates the inherent limitations of the
Cloud Computing model:

\begin{itemize}
\item The externalization of private applications/data often faces legal issues
that restrain companies from outsourcing them on external infrastructures,
especially when located in other countries. 
\item The overhead implied by the unavoidable use of the Internet to reach
distant platforms is wasteful and costly in several situations: Deploying a
broadcasting service of local events or an online service to order pizzas at the
edge of the polar circle, for instance, leads to important overheads since
most of the users are \emph{a priori} located in the neighborhood of the
event/the pizzeria.  
\item The connectivity to the application/data cannot be ensured by centralized
dedicated centers, especially if they are located in a similar geographical
zone. The only way to ensure disaster recovery is to leverage distinct
sites.\footnote{``Amazon outages – lessons learned'',
\href{http://gigaom.com/cloud/amazon-outages-lessons-learned/}{http://gigaom.com/cloud/amazon-outages-lessons-learned/}
(valid on  Nov 2013, the 30\textsuperscript{th}).} 
\end{itemize}

The two first points could be partially tackled by hybrid or federated Cloud
solutions~\cite{armbrust:2010}, that aim at extending
the resources available on one Cloud with those of another one; however, the third 
one requires a disruptive change in
the way UC resources are managed.
%Deploying a local events broadcasting service or an
%online service to order pizza at the edge of the polar circle for instance, leads to an important overhead
%in terms of energy footprint, network exchanges as well as latency since it can be assumed
%that a vast majority of the users are located in the neighborhood of the event/the
%pizzeria.

Another issue is that, according to some projections of a recent IEEE
report~\cite{ieeenetreport:2012}, the network traffic continues to double roughly each
year. Consequently, bringing the IT services closer to the end-users is becoming crucial to limit
the energy impact of these exchanges and to save the bandwidth of some links. Similarly,
this notion of locality is also critical for the adoption of the UC model by applications
that need to deal with a large amount of data as getting them in and out actual UC
infrastructures may significantly impact the global performance~\cite{Fos11}. 

The concept of micro/nano DCs at the edge of the 
backbone~\cite{greenberg:sigcomm09} may be seen as a complementary solution to
hybrid platforms in order to reduce the overhead of network exchanges.
However, operating multiple small DCs breaks in somehow the idea of
mutualization in terms of physical resources and administration simplicity, making this approach questionable.
%Moreover,  the number of such micro/nano DCs will remain limited and the question of where and
%how federating a large number of such facilities are still not solved.

\subsection{Ubiquitous and Oversized Network Backbones}
One way to partially resolve the mutualization concern enlightened 
by the defenders of large-scale DCs is to directly deploy the concept of micro/nano DCs upon the Internet backbone.
People are (and will be) more and more
surrounded by computing resources, especially those in charge of
interconnecting all IT equipments. Even though these small and medium-sized
facilities include resources that are barely used~\cite{Andrew:2003,
Benson:2010}, they can hardly be removed (\textit{e.g.} routers).  Considering
this important aspect, we claim that a new generation of UC platforms can be
delivered by leveraging existing network centers, starting from the core nodes of the backbone to the
different network access points in charge of interconnecting public and private
institutions.  By such a mean, network and UC providers would be able to
mutualize resources that are mandatory to operate network/data centers while
delivering widely distributed UC platforms that can better match the
geographical dispersal of users. 
%
% TODO: AL -> ACO, please introduce this point latter in the chapter
% As a consequence, several initiatives started investigating how they could be
%better leveraged to support the requirements and constraints of current IT
%usages.  The concept of \emph{data furnaces} \cite{liu:hotcloud11} is one of
%the promising idea that seeks to mitigate the cost of operating
%network/computing resources by using them as a source of heat inside public
%buildings such as hospitals or universities. 
%
Figure~\ref{fig:renater} allows to better capture the advantages of such a proposal.
%\ftodo[FQ$\rightarrow$ALL]{Nous n'avons pas encore introduit la contribution à ce stade $\rightarrow$ enlever cette phrase~?}
% We did, cf. above : WE CLAIM
It
 shows a snapshot of the network weather
map of RENATER\footnote{\href{http://www.renater.fr}{http://www.renater.fr}}, the backbone dedicated to universities and research
institutions in France. It reveals several important points: 
\begin{figure}[b]
\vspace*{-.3cm}
\includegraphics[width=10cm]{./FIGS/renater.png}
\centering\caption{The RENATER Weather Map on May 2013, the 27th, around 4PM.
Each red square corresponds to a particular point of presence (PoP) of the network. The map is available in real-time
at: \href{http://www.renater.fr/raccourci}{http://www.renater.fr/raccourci}}
\label{fig:renater}
\vspace*{-.3cm}
\end{figure}


\begin{itemize} 
\item As mentioned before, most of the resources are under-used (only two links are used between 45\% and 55\%, a few between 25\% and 40\% and the majority below the threshold of 25\%). 
\item The backbone was deployed and is renewed to match the demand: The density of
points of presence~(PoP, \ie a small or medium-sized network centers) as well as the bandwidth of each link are more important on the edge of large cities such as Paris, Lyon or
Marseille. 
\item The backbone was designed to avoid disconnections, since 95\% of the PoPs can be reached by at least two distinct routes.
\end{itemize}


\ftodo[AL -> ALL]{It might make sense to talk of congestion effects that we can see in Marseille and Paris, the two path to the rest of Internet.}

\subsection{Locality-based Utility Computing}

%This chapter aims at introducing a new generation of UC platforms that can be
%seen somehow as an extension of the concept of micro DCs. The main change is to
%consider locality as a key point of UC services and to leverage facilities
%composing the internet backbone.  %Instead of building and deploying dedicated
%facilities, we claim that next UC
%infrastructures should be tightly coupled with any facilities available through
%the Internet, starting from the core routers of the backbone, the different
%network access points and any small and medium-size computing infrastructures
%that may be provisioned by public and private institutions. 

% Although it involves radical changes in the way
%physical and virtual resources are managed, locating and operating computing
%power and data on
%facilities close to the end-users will deliver highly efficient
%and sustainable UC services, resolving inherent limitations of the cloud computing model leveraging large-scale DCs. 

This chapter aims at introducing locality-based UC infrastructures, a new
generation of UC platforms that solves inherent limitations of the Cloud Computing 
paradigm relying on large-scale DCs. Although it involves radical changes in
the way physical and virtual resources are managed,  leveraging network centers
is a promising way to deliver highly efficient and sustainable UC services. 

From the physical point of view, network backbones 
%such as National Research and Educational Networks (NRENs) 
provide appropriate infrastructures, \ie, reliable and efficient enough to operate UC
resources spread across the different PoPs. Ideally, UC resources would be able to
directly
take advantage of computation cycles available on network active devices, \textit{i.e.} those
in charge of routing packets. However, leveraging network resources to make external
computations may lead to important security concerns. Hence, we propose to extend each
POP with a number of servers dedicated to VM hosting.
Because it is natural to assume that the network traffic and UC demands are proportional, larger network
centers will be completed by more UC resources than the smaller ones. Moreover, by deploying
UC services on relevant PoPs, a LUC infrastructure will be able to natively confine
network exchanges to a minimal scope, minimizing both the energy footprint of the network, 
the impact on latency and the congestion phenomena that may occur on critical paths (for instance Paris and Marseille on RENATER). 

From the software point of view, the main challenge is to design a complete distributed
system in charge of turning a complex and diverse network of resources into a collection
of abstracted computing facilities that is both reliable and easy to operate.

\begin{svgraybox}
The \emph{LUC Operating System}, an advanced system being able to operate 
many UC resources distributed on distinct sites  would enable 
Internet service providers~(ISPs) and other institutions in
charge of operating a network backbone to build an extreme-scale
LUC infrastructure with a limited additional cost. Instead of redeploying a
complete installation, they will be able to leverage IT resources and
specific devices such as computer room air conditioning units, inverters or
redundant power supplies already present in each center of their
backbone. 
\end{svgraybox}


\medskip
%This chapter  describes  how such a new
%generation of highly efficient and sustainable UC can emerge through an integrated
%system, \ie the \emph{LUC Operating System}, leveraging advanced and P2P system mechanisms.

In addition to considering \emph{locality} as a primary concern, the novelty of the LUC OS
proposal is to consider the VM as the basic object it manipulates.  Unlike existing
research on distributed operating systems designed around the process concept, a LUC OS will manipulate VMs throughout a federation of widely distributed
physical machines. Virtualization technologies abstract out hardware heterogeneity, and allow
transparent deployment, preemption, and migration of virtual
environments~(VEs), \ie a set of interconnected VMs.
By dramatically increasing the flexibility of resource management, virtualization 
allows to leverage state-of-the-art results from other distributed
systems areas such as autonomous and decentralized systems.  
Our goal is to build a system that allows end-users to launch VEs over a
distributed infrastructure as simply as they launch processes on a
local machine, \ie  without the burden of dealing with resources
availability or location.

%\paragraph{Chapter Outline.} 
Section~\ref{sec:challenges} describes the key objectives of a LUC OS and the associated challenges. 
Section~\ref{sec:background} explains why our vision differs from actual and previous UC solutions. In
Section~\ref{sec:archi}, we present how such a unified system may be designed
by delivering the premises of the \discovery system, an agent-based system
enabling distributed and cooperative management of virtual environments over a
large-scale distributed infrastructure.
Future work as well as opportunities  are addressed in Section~\ref{sec:future}. Finally Section~\ref{sec:conclusion} concludes this chapter. 

%%%
\section{Overall Vision and Major Challenges\label{sec:challenges}}

Similarly to traditional operating systems~(OSes), a LUC OS will be composed of
many mechanisms. Trying to identify all of them and establishing how they
interact is an on-going work (see Section~\ref{sec:archi}). However,
%having in
%mind the goal of delivering a unified system in charge of operating a complex and diverse
%infrastructure, and transforming it into a LUC platform,
we have pointed out the following
key objectives to be considered when designing a LUC OS:

\begin{itemize} 
\item Scalability: a LUC OS must be able to manage hundreds of
  thousands of virtual machines~(VMs) running on thousands of 
  geographically distributed computing resources.  These resources are small or
  medium-sized computing facilities and may become highly volatile according to the network disconnections.  
\item Reactivity: To deal with the dynamicity of the infrastructure, a LUC OS
  should swiftly handle events that require performing particular
  operations, either on virtual or on physical resources. This has to be done with the
  objective of maximizing the system utilization while meeting the quality of service~(QoS) expectations of VEs. 
  Some examples of operations that should be performed as fast as possible include (i)~the reconfiguration
  of VEs over distributed resources, sometimes spread across wide area networks, or (ii)~the migration of VMs, 
  while preserving their active connections.
\item Resiliency: In addition to the inherent dynamicity of the
  infrastructure, failures and faults should be considered as the norm rather than the
exception at such a scale. The goal is therefore to transparently leverage the
underlying infrastructure redundancy to (i)~allow the LUC OS to keep
working despite node failures and network disconnections (LUC OS robustness) and to (ii)~provide
snapshotting as well as high availability mechanisms for VEs (VM Robustness).
\item Sustainability: Although the LUC approach would reduce the energy
footprint of UC services by minimizing the cost of the network, 
it is important to go one
step further by considering energy aspects at each level of a LUC OS
and propose advanced mechanisms in charge of making an optimal usage of each source of energy. 
%Minimizing the energy footprint is a
%  transversal concern that has to be considered at each level of the
%  design of \discovery.
  To achieve such an objective, the LUC OS should take account of data related to the
  energy consumption of the VEs and the computing resources, as well as the environmental
  conditions (computer room air conditioning unit, location of the site, etc.).
\item \textbf{Security and Privacy:} Similarly to resiliency, security, and privacy issues
  affect the LUC OS itself and the VEs running on it. Regarding the LUC OS, the goals are
  to (i)~create trust relationships between different locations, (ii)~secure the
  peer-to-peer layers, (iii)~include security and privacy decision and enforcement points
  in the LUC OS and (iv)~make them collaborate through the secured peer-to-peer layers to
  provide end-to-end security and privacy.
%  at different layers and locations to provide a end-to-end and in-depth security enforcement.
  Regarding the VEs, users should be able to express their requirements in terms of
  security and privacy; these requirements would then be enforced by the LUC OS.

% \ftodo[AL/JP]{More on privacy? Only security is mentioned.}

\end{itemize}

In addition to the aforementioned objectives, working on a virtual infrastructure requires
to deal with the management of VM images. Managing VM images in a distributed way across a
wide area network~(WAN) is a real challenge that will require to adapt state-of-the-art
techniques related to replication and deduplication. Also, the LUC OS must take into
account VM images location, for instance (i)~to allocate the right resources to a VE or
(ii)~to prefetch VM images, to improve deployment performance or VM relocations.

Finally, one last scientific and
technical challenge is the lack of a global view of the infrastructure.  Maintaining a
global view would indeed limit the scalability of the LUC OS, which is inconsistent with
our objective to manage large-scale geographically distributed systems.  Therefore, we
claim that the LUC OS should rely on decentralized and autonomous mechanisms, that can
match and adapt to the volatile topology of the infrastructure.  Several decentralized
mechanisms are already used in production on large-scale systems; for instance, Amazon
relies on the Dynamo service~\cite{decandia:2007} to create distributed indexes and
recover from data inconsistencies; moreover, Facebook uses Cassandra~\cite{lakshman:2010},
a massive scale structured store that leverages peer-to-peer techniques.
%
%Among the numerous scientific and technical challenges that should be addressed, 
%the lack of a global view of the system introduces a lot
%of complexity. In order to tackle it while addressing the above-mentioned
%challenges, we claim that internal mechanisms of a LUC OS should be based
%on decentralized mechanisms specifically designed for it.
%% the latest contributions in distributed and
%% cooperative algorithms such as gossip-based approaches and self-* techniques.
%These techniques should provide mechanisms which are fully decentralized and
%autonomous, so to allow self-adapting control and monitoring of complex
%large-scale systems. Simple locality-based actions by each of the entities
%composing the system can lead to the global emergence of complex and
%sophisticated behaviors, such as the self-optimization of resource allocation,
%or the creation of decentralized directories. These techniques are starting to
%be used in well-known large systems. As an example, the Amazon website relies on
%its decentralized Dynamo service~\cite{decandia:2007} to create largely distributed indexes and recover from data
%inconsistencies. Facebook’s Cassandra massive scale structured
%store~\cite{lakshman:2010} also leverages P2P techniques for its core
%operation.
%
In a LUC OS, decentralized and self-organizing overlays will enable to maintain the
information about the current state of both virtual and physical resources, their
characteristics and availabilities. Such information is mandatory to build higher-level
mechanisms ensuring the correct execution of VEs throughout the whole infrastructure.
%However, it is worth noting that simultaneous local actions can lead to the global
%emergence of complex behaviors.




\subsection{DVMS}

%\subsubsection{Overview.}
DVMS~\cite{quesnel:ispa2013,quesnel:cpe2012}
(Distributed Virtual Machine Scheduler) is a framework that schedule VMs
cooperatively and dynamically in large scale distributed
systems. It is deployed as a set of agents that are organized following a ring
topology and that cooperate with one another to guarantee that VM demands are satisfied during their executions. 
Concretely, when a node cannot guarantee the QoS for its hosted VMs or when it is
under-utilized, it starts an iterative scheduling procedure~(ISP) by querying
its neighbor to find a better placement ; it thus becomes the initiator of the ISP.
If the request cannot be satisfied by the neighbor, it is forwarded to the
following free one until the ISP succeeds. When a viable mapping has been
found, the leader (\ie the last peer that has taken part to the ISP) reconfigures 
the system by performing adequate VM migrations.
Such an approach allows each ISP to send requests only to a minimal number of
nodes and even though an ISP can reserve all nodes if the corresponding problem
is particularly hard to solve (guaranteeing thus that a solution will always be
found it it exits), experiments have shown that in most cases ISPs involve only
few nodes.  Moreover, the DVMS proposal allows several ISPs to occur
independently at the same moment throughout the infrastructure; in other words,
scheduling is performed on partitions of the system that are created
dynamically, which significantly improves the reactivity of the system.  To
prevent conflicts that could occur if several ISPs performed concurrent
operations on the same PMs or VMs, it should be noted that PMs are reserved for
exclusive use by a single ISP. 

An example involving three partitions is shown on Figure~\ref{fig:isp}; in
particular, we can see the growth of partition~1 between two steps. 
Explaining in details the notion of ``first out'' is beyond the scope of this article but readers can consider that the ``first out'' relation enables
to handle communications efficiently, as each node involved in a partition
can forward a request directly to the first node outside its partition~\cite{quesnel:cpe2012}.
\begin{figure}[h!]
  \centering
  \includegraphics[width=0.9\linewidth]{Figures/resourceAcquisition-standard.pdf}
  \caption{Solving three problems simultaneously and independently with DVMS}%
\small{The ring has been matched on top of three sites}
  \label{fig:isp}%
\end{figure}

In addition to formally prove the correctness of DVMS, the first version of the prototype
has been validated at large scale (up to 80k VMs and 8k nodes by means of simulations and up to 4.7k VMs and 470 nodes by means of experiments on the Grid'5000
testbed~\cite{quesnel:ispa2013}).

As discussed earlier, one limitation of this approach is related to its ring topology that prevents to take into account the network topology. 
%\subsubsection{Limitations of The Ring}
%
%Even though  DVMS can be deployed across several sites, it performs better on a
%cluster.
%
%The reason is simple.
%
%The ring is built without taking account of the network  topology; 
In other words, if the ISP strategy enables to limit the size of one partition to a minimal number of nodes, these nodes are selected without considering 
the network conditions at the time the ISP starts; leading to inefficient situations where VM migrations occur between two nodes that are far from each other, 
which lasts longer than a migration between two close nodes. Obviously the ring can
be built in order to limit the distance between peers globally (\ie peers of
the same region/area would be grouped together as illustrated on Figure \ref{fig:isp}). However, in such a case at least two nodes of each group are directly connected to 
two far nodes. 
%
To sum up, the DVMS proposal lacks of a topology that can consider locality properties of a multi-sites infrastructure. 

\AL[CD/MB]{Can we add one sentence to make the transition to the next section: something for instance that explains that a hierarchy of rings will not tackle the aforementioned concern.}
% Autres limitations :
% -absence de tolérance aux pannes
% -peu d'événements gérés (surcharge d'un noeud)
% -ne prend pas en compte les liens entre VMS

\subsection{Overlay networks and locality}

Taking the locality into account in the construction of overlay networks was
initially proposed in the Pastry overlay network~\cite{pastry}. In order to
reduce the latency of the routing process, each node is given the opportunity to
choose the closest nodes to fill its routing table. Learning the existence of
new nodes relies on periodic exchange of the references of nodes known by nodes.


Le m?me concept a ?t? propos? dans les r?seaux logiques non structur?s afin de
permettre ? chaque n\oe ud de d?couvrir des n\oe uds du syst?mes les plus
\emph{proches}. La notion de proximit? peut recouvrir toute m?trique transitive
entre deux n\oe uds, en particulier le temps de latence entre les n\oe
uds~\cite{refquivabienmarindoittrouver}.

Le protocole Vivaldi~\cite{dabek:2001:sigcomm04}, sur lequel notre r?seau logique est fond?,
a une approche particuli?re. En effet, il fournit ? chaque n\oe ud des
coordonn?es dans un espace multi-dimensionnel refl?tant sa position dans le
r?seau physique. Initialement, chaque n\oe ud prend une position al?atoire de
l'espace, et choisit un petit sous-ensemble de n\oe uds. Puis, il se rapproche
dans l'espace, des n\oe uds avec lesquels il a une faible latence et s'?loigne
dans le cas inverse. Vivaldi ne permet donc pas de conna?tre les n\oe uds qui
lui sont proches dans le r?seau, mais de les reconna?tre via leurs coordonn?es.

Les approches pr?c?dentes maintiennent constamment la connaissance des n\oe udes
proches afin de fournir le meilleur n\oe ud possible, au co?t de communications
p?riodiques (ind?pendamment de la quantit? de requ?tes effectives.) Notre
approche se distingue par une approche paresseuse consistant ? rechercher des
n\oe uds proches (en s'appuyant sur les coordonn?es Vivaldi) lors des requ?tes,
adaptant ainsi la qualit? de la r?ponse ? la fr?quence des requ?tes.


%%%
\section{Premises of a LUC OS: The \discovery Proposal\label{sec:archi}}

In this section, we propose to go one step further by discussing preliminary
investigations around the design and the implementation of a first LUC OS
proposal: the \discovery system~(DIStributed and COoperative framework to manage
Virtual EnviRonments autonomouslY). We draw the premises of the \discovery
system by emphasizing some of the challenges as well as some research directions
to solve them. Finally, we give some details regarding the prototype that is
under development and how we are going to evaluate it.  

\subsection{Overview}

The \discovery system relies on a multi-agent peer-to-peer system deployed on
each physical resource composing the LUC infrastructure. Agents are autonomous
entities that collaborate with one another to efficiently use the LUC resources. In our context,
efficiency means that a good trade-off is found between users'
expectations, reliability, reactivity and availability,
while limiting the energy consumption of the system and providing
scalability. 
%To reduce the management complexity as well as the design and the
%implementation of the critical mechanisms, we strongly
%support the use of micro-kernel concepts. Such an approach should enable us to design
%and implement services at higher level while leveraging peer-to-peer mechanisms
%at the lower ones.
%Furthermore, to address the different objectives and reduce
%the management complexity, we also underline that self-* properties should be
%present at every level of the system.  We think that relying on a multi-agent
%peer-to-peer system is the best solution to cope with the scale as well as the
%network disconnections that may create temporary partitions in a LUC platform.

In \discovery, each agent has two purposes: (i)~maintaining a knowledge base on the
composition of the LUC platform and (ii)~ensuring the correct execution of VEs. 
%Concretely, the knowledge base will consist of overlays that will be used 
%for the autonomous management of the
%VEs life cycle.
This includes the configuration, deployment and monitoring of
VEs as well as the dynamic allocation or relocation of VMs to adapt to changes
in VEs requirements and physical resources availability. To this end, agents
will rely on dedicated mechanisms related to: 

\begin{itemize}
\item The localization and monitoring of physical resources, 
\item The management of VEs, 
\item The management of VM images, 
\item Reliability,
\item Security and privacy.
\end{itemize}

\subsection{Resource Localization and Monitoring Mechanisms\label{ssec:p2p}}

Keeping in mind that \discovery should be designed in a fully decentralized
fashion, its mechanisms should be built on top of an overlay network able to
abstract out changes that occur at the physical level. The specific requirements
of this platform will lead to the development of a novel kind of overlay
networks, based on locality and a minimalistic design.
%
More concretely, the first step is to design, at the lowest level, an overlay
layer intended to hide the details of the physical routes and computing
utilities, while satisfying some basic requirements such as locality and
availability. This overlay needs to enable the communications between any two
nodes in the platform. While overlay computing has been extensively studied over
the last decade, we emphasize here on minimalism, and especially on one key
feature to implement a LUC OS:  retrieving nodes that are geographically close
to a given departure node.

\subsubsection*{Giving Nodes a Position}

The initial configuration of the physical network can take an arbitrary
shape. We choose to rely on the Vivaldi
protocol~\cite{dabek:2001:sigcomm04}. Vivaldi is a distributed algorithm
assigning coordinates in the plane to nodes of a distributed system. Each node
is equipped with a \emph{view} of the network, \emph{i.e.}, a set of nodes it
knows. This view is initially assumed as random. Coordinates obtained by a node
reflects its \emph{position} in the network, \emph{i.e.}, close nodes in the
network are given close coordinates in the plane. To achieve this, each node
periodically checks the round trip time between itself and another node
(randomly chosen among nodes in its view) and adapts its distance (by changing
its coordinates) with this node in the plane accordingly. See
Figure~\ref{fig:vivaldi_before} and Figure~\ref{fig:vivaldi_after} for an
illustration of 4 nodes~(A, B, C and D) moving according to the Vivaldi
protocol. 
%
A globally accurate positioning of nodes can be
obtained if nodes have a few long-distance nodes in their
view~\cite{dabek:2001:sigcomm04}. These long distance links can be easily
maintained by means of a simple gossip protocol.

\begin{figure}[!b]
	\vspace*{-.3cm}
  \begin{minipage}[c]{.45\linewidth}
   \hspace*{-0.5cm}
      	\centering \includegraphics[width=3.4cm]{./FIGS/vivaldi_before.pdf}

   \hspace*{0.5cm}
		\caption{Vivaldi plot before updating positions. Each node pings other nodes. Each node maintains a map of distance.}
\label{fig:vivaldi_before}
   \end{minipage}
\hspace*{0.6cm}
   \begin{minipage}[c]{.45\linewidth}
   	\centering \includegraphics[width=3.4cm]{./FIGS/vivaldi_after.pdf}
		\caption{Vivaldi plot after updating positions. The computed
                  positions of other nodes have been updated.}
		\label{fig:vivaldi_after} 
  \end{minipage} \hfill
\end{figure}


\subsubsection*{Searching for Close Nodes}

Once the map is achieved (each node knows its coordinates), we are able to decide
whether two nodes are \emph{close} by calculating their distance. However, the view of
each node does not \emph{a priori} contain its closest nodes. Therefore, we need additional
mechanisms to locate a set of nodes that are close to a given initial node --
Vivaldi gives a \emph{location} to each node, but not a neighborhood. To achieve
this, we use a modified distributed version of the classic Dijkstra's algorithms
used to find the shortest path between two nodes in a graph. The goal is to
build a \emph{spiral}\footnote{The term \emph{spiral} is here a misuse of
language, since the graph actually drawn in the plane
might contain crossing edges. The only guarantee is that when following the
path constructed, the nodes are always further from the initial node.}
interconnecting the nodes in the plane that are the closest ones to a given initial
node.

Let us consider that our initial point is a node called $I$. The first step is
to find a node to build a two-node spiral with $I$. Such a node is sought in the
view of $I$ by selecting the node, say $S$, having the smallest distance with
$I$. $I$ then sends its view to $S$, $I$ stores $S$ as its successor in the
spiral, and $S$ adds $I$ as its predecessor in the spiral. Then $I$ forwards its
view to $S$. $S$ creates a new view by keeping the $n$ nodes which are the
closest to $I$ in the views of $I$ and $S$. This last view is then referred to
as the \emph{spiral view} and is intended to contain a set of nodes among which
to find the next step of the spiral. Then $S$ restarts the same process: Among
the spiral view, it chooses the node with the smallest distance to $I$, say
$S'$, and adds it in the spiral -- $S$ becomes the predecessor of $S'$ and $S'$
becomes the successor of $S$. Then, the spiral view is sent to $S'$ which
updates it with the nodes it has in its own view. The process is repeated until
we consider that enough nodes have been gathered (a parameter sent by the
application).

Note that one risk is to be blocked by having a spiral view containing only
nodes that are already in the spiral, leading to the impossibility to build the
spiral further. However, this problem can be easily addressed by forcing the
presence of a few long distance nodes whenever it is updated.

\subsubsection*{Learning}

Applying the protocol described above, the quality of the spiral is
questionable in the sense that the nodes that are actually close to the starting
node $s$ may not be included. The only property ensured is that one step
forward on the built path always takes us further from the initial node.

To improve the \emph{quality} of the spiral, \emph{i.e.}, reduce the average
distance between the nodes it comprises and the initial node, we add a learning
mechanism coming with no extra communication cost: when a node is contacted for
becoming the next node in one spiral, and receives the associated spiral view,
it can also keep the nodes that are the closest to itself, thus potentially
increasing the quality of a future spiral construction.

\subsubsection*{Routing}

In the context of a LUC infrastructure, one crucial feature is to be able to
locate an existing VM. Having the same strategy consisting in improving the
performance of the overlay based on the activity of the application, we envision a
routing mechanism which will be improved by past routing requests. By means of the
spiral mechanism, a node is able to contact its neighboring
nodes to start routing a message.

This initial routing mechanism can be very expensive, as the number of hops can
be linear in the size of the network. However, from previous communications, a
node is able to memorize long links to different locations of the
network. Consequently, from each routing request, the source of the request and
each node on the path to the destination are able to learn long links, which
will significantly reduce the number of hops of future requests. We are
currently studying the amount of requests needed to get close to a logarithmic
routing complexity. More generally, we are working on the estimation if the
activity of the application required to (i)~guarantee the constant efficiency of
the overlay and to (ii)~converge, starting from a random configuration, to a
fully-efficient overlay network.

%\ftodo[CT]{TODO}

% Each node doing so in parallel, a number of clusters are created inside which we
% can ensure a certain level of locality -- all nodes inside this group can
% communicate with each other very efficiently. Given the number of nodes in these
% groups, the inner topology of a group can either rely on very simple graphs,
% such as rings, or more connected graphs, to accelerate the dissemination and
% retrieval of information in the group. Note that, still based on simple
% gossiping techniques, such graphs can be easily maintained as the network's
% conditions change. For instance, if a link becomes overloaded, the other nodes
% will react to this change by removing nodes with which they communicate through
% this link from their local group. Such groups are exemplified on
% Figure~\ref{fig:renater_overlay} (for the west part of the platform).

% \begin{figure}[htbp]
% \includegraphics[width=12cm]{./FIGS/renater_overlay.png}
% \caption{Overlay local groups on top of the RENATER platform.\label{fig:renater_overlay}}
% \end{figure}

% As we can see on Figure~\ref{fig:renater_overlay}, it follows from the way the
% overlay is built that some nodes may be a member of several groups. The Nantes
% site for instance is part of three of these groups, given its
% physical position. More generally, the overlay will take the shape of a set of
% groups with some \emph{bridges} between the groups. Note that several
% \emph{bridges} can interconnect two local groups. Thus, any request first goes
% through local nodes allowing for its local processing, and avoiding the need for
% global coordination mechanisms.

% To be able to localize a particular VM, the system has to be able to route a
% request from any node to any other node. This functionality is the classic
% problem of P2P overlays, which is traditionally solved in structured overlay networks by
% maintaining a routing table on each node, and by a flooding mechanism or with
% random walks in unstructured overlays. These techniques are usually designed for
% very large scale networks with different guarantees and costs. Here, given the
% particular ``intermediate'' scale of the platform, and its specific
% requirements, we believe that these existing techniques are too
% \emph{powerful}. As we want to design a minimalistic overlay, there is room to
% try to design a new routing technique specifically fitting our
% requirements. The aim is then to maintain, in all groups, information about the
% distance between this group and \emph{close} groups in terms of number of bridge nodes
% to go through. Hence, the system will be able to route quickly between \emph{close}
% groups. The routing of requests between \emph{far} groups will be based on a random
% decision when no information are available, but \emph{oriented} by the aim of
% going away from the request's source.

% This overlay will provide the basic building block of the platform, on which will
% rely higher level overlays and functionalities, which are described in the
% following sections.

% \subsubsection*{Upper-Layer overlays}

% \ftodo[CT]{To be completed...}


% \ftodo[AL]{To be completed by Cedric and Marin}
% \ftodo[AL]{C/P from the proposal : 
% self-organizing overlay construction mechanisms based on gossip and epidemic
% techniques that will link the \discovery resources in navigable graphs onto
% which requests for VE can be routed and allocated. It will investigate
% mechanisms to quickly adapt these graphs to changing conditions and
% infrastructure, and mechanisms to monitor the adequacy of previous requests
% solutions to these changing conditions as these occur.}

\subsection{VEs Management Mechanisms}
\label{ssec:vem}
In the \discovery system, we define a VE as a set of VMs that may have specific
requirements in terms of hardware, software and also in terms of placement:~
For instance, some VMs must be on the same node/site to cope with performance objectives while
others should not be collocated to ensure
high-availability criteria~\cite{hermenier:2013}.
As operations on a VE may occur in any place from any location, each agent should provide the capability
to configure and start a VE, to suspend/resume/stop it, to relocate some of its VMs if need be or simply to retrieve the location of a particular VE. 
Most of these mechanisms are provided by current UC platforms. However, as mentioned before, they
should be revisited and leverage peer-to-peer mechanisms to correctly run on the infrastructure we target
(\textit{i.e.} in terms of scalability, resiliency and reliability).
%To this aim, the \discovery system relies on the aforementioned peer-to-peer mechanisms. 

As a first example, placing the VMs of a VE requires to be able to find available nodes that
fulfill the VM needs (in terms of resource requirements as well as placement
constraints). Such a placement can start locally, close to the client
application requesting it, \textit{i.e.}, in its local group. If no such node is
found, a simple navigation ensures that the request will encounter a bridge,
leading to the exploration of further nodes. This navigation goes
on until an adequate node is found.
A similar process is performed by the mechanism in charge of  dynamically
controlling and adapting the placement of VEs during their lifetime.  For instance, 
to ensure the particular needs of a VM, it can be necessary to relocate other VMs.
According to the predefined constraints of VEs, some VMs might be
relocated on far nodes while others would prefer to be suspended.  Such a
mechanism has been deeply studied in the DVMS
framework~\cite{dvms:wiki,quesnel:2012}. DVMS~(Distributed Virtual
Machine Scheduler) is able to dynamically schedule a significant number of VMs
throughout a large-scale distributed infrastructure while guaranteeing VM
resource expectations.  

A second example regards the networking configuration of VEs.
Although it might look simple, assigning the right IP to
each VM as well as maintaining the intra-connectivity of a VE becomes a bit more complex than in
the case of a single network domain, \textit{i.e.} a single site deployment.
%
Keeping in mind that a LUC infrastructure is
by definition spread WANwide, a VE can be hosted between distinct network
domains during its lifetime. No solution has been chosen yet. 
Our first investigations led us to leverage techniques
such as the IP over P2P project~\cite{ganguly:2006}. However,
software-defined networking becomes more and more important; investigating proposals such as
the Open vSwitch project~\cite{pfaff:2009} looks promising to solve such an issue.
%

\subsection{VM Images Management}

In a LUC infrastructure, VM images could be deployed in any place from any
other location. However, being in a decentralized, large-scale, heterogeneous and widely spread
environment makes the management of VM images more difficult than with
conventional centralized repositories.  
At coarse grain, the management of the VM images should be (i)~consistent
with regard to the location of each VM in the \discovery infrastructure and
(ii)~reachable in case of node failures or network disconnections.
%
The envisioned mechanisms to manage VM images have been
classified into two categories.
%
First, some mechanisms are required to efficiently
upload VM images and replicate them across many nodes, to ensure
efficiency as well as reliability.  Second, other mechanisms 
are needed to schedule VM image
transfers. Advanced policies are important to improve the efficiency of each
transfer that may occur either during the first deployment of a VM or during its relocations. 

Regarding storage and replication mechanisms,  an analysis of an IBM Cloud concludes
that a fully distributed approach using peer-to-peer technology is not the best choice to manage VM images, since the
number of instances of the same VM image is rather small~\cite{peng:2012}. However, central or
hierarchical solutions are not suited for the infrastructure we target.
Consequently, an improved peer-to-peer solution working with replicas and
deduplication has to be investigated, to provide more
reliability, speed, and scalability to the system. For example, analyzing
different VM images shows that at least 30\% of the image is shared between
different VMs~\cite{jin:systor2009}. This 30\% can become a 30\% reduction in space, or a
30\% increase in reliability or in transfer speed. Depending on the
situation, we should decide to go from one scenario to another. 
%Finally, the number of replicas and deduplicated data should be dynamically balanced.

Regarding the scheduling mechanisms, a study showed that VM boot time can increase
from 10 to 240 seconds when multiple VMs running I/O intensive tasks use the
same storage system~\cite{tan:2008}. Some actions, like providing the
image chunks needed to boot first~\cite{tang:2011}, defining a new
image format, and pausing the rest of the I/O operations, can provide a
performance boost and limit the overhead that is still observed in commercial
Clouds~\cite{mao:2012}. 
%These actions should also take into account
%power-like metrics in order to reduce the energy consumption of data transfers.
%According to (Preist et Shabajee Nov 2010) the cost to transmit 1 MB can be of
%4Wh. Considering the size of VM images, any improvement aiming at reducing
%data movement will make a big difference.

More generally, the amount of data linked with VM images is significant.
Actions involving data  should be aware of consequences on metrics like
(but not limited to): energy efficiency, reliability, proximity, bandwidth and
hardware usage. The scheduler could also anticipate actions, for instance moving images when
the load is low or the energy is cheap.

%% \subsection{Reliability Mechanisms}
%% %These mechanisms should ensure  reliability and high availability of the
%% %\discovery system despite the scale and dynamicity of the underlying physical
%% %infrastructure.   
%% By nature, a LUC is a highly distributed platform where 
%% node and network failures will be much more frequent than in
%% actual UC platforms.  Furthermore, since resources could be located anywhere,
%% the expected mean time to repair failed equipments might be much larger than in
%% other platforms. For all these reasons, a set of dedicated mechanisms should be designed in order
%% to provide a fully transparent failure management with minimum downtime. 
%% %
%% To this aim, the \discovery system should include, first,  mechanisms in charge of its own reliability. 
%% Such mechanisms are required to avoid losing or corrupting important information
%% regarding the state of the system.  Of course, handling all kinds of failures
%% and implementing a fully resilient operating system is a complex task. Hence,
%% we propose to consider in a first time a crash-stop failure model. In other words, the
%% \discovery system should be able to autonomously restart any service by
%% relocating it on an healthy agent each time it is mandatory.  This implies to
%% define a common design pattern, \textit{i.e.} a set of recommendations, that each
%% service should follow to ensure such characteristics. Besides, in order to
%% ensure that such a pattern may be applied to stateful services, the system can
%% require a Cassandra like system \cite{lakshman:2010} that provides reliable and highly available
%% storage for critical system states. Retrieving the information related to one
%% service need to rely on the kind of mechanisms described in Section~\ref{ssec:p2p}.

%% %
%% In addition to be robust enough, the \discovery system should ensure the
%% reliable execution of virtual environments. The first mechanisms consists in
%% using snapshotting capabilities delivering by virtualization technologies.
%% Concretely, each internal states of a VM should be periodically saved on a
%% persistent storage.  When a crash occurs on a VM, its associated VE can be
%% restarted from the latest consistent states, \textit{i.e.} all VMs of the VE will be
%% resume for their latest snapshot.  As for the other mechanisms, performing VM
%% snapshotting in a large-scale, heterogeneous and widely spread environment is a
%% challenging task. However, we believe that aforementioned mechanisms in charge
%% of the VM images as well as recent proposals \cite{nicolae:2011} might enable
%% to provide such a feature. 
%% Although VM snapshotting provides a first level of reliability, it is not
%% sufficient to ensure high availability of the VE. More advanced mechanisms must
%% be proposed.  Our idea is to include mechanisms based on primary-backup
%% replication techniques. 
%% The basic principle is to have one active replica of the VM (the primary)
%% sending state updates to the other replicas (the backups) periodically. If the
%% primary fails, one of the backup can resume the execution transparently for the
%% outside world. Furthermore since the entire VM is replicated, applications can
%% be run unmodified.  Solutions to replicate VMs inside a cluster have been
%% proposed. However efficiently replicating VMs over a WAN is a huge challenge.
%% Limiting the size of the backup updates\cite{rajagopalan:2012}, and
%% reducing the impact of the required synchronizations on the execution of the
%% primary \cite{gerofi:2012}  are research directions to be further
%% studied. A better understanding of the parts of a VM that really need to be
%% updated is required. It might require to trade transparency for performance by
%% allowing latency-sensitive applications to define which part of their state has
%% to be updated.

%% \ftodo[AL]{Integrate the P2P aspect into the reliability section}
%% \subsubsection*{Monitoring Example}

%% Another key use of this low-level overlay is proactive replication of VMs,
%% keeping in mind that two identical VMs should be placed in relatively distant
%% nodes, for fault-tolerance reasons (close nodes have a high probability to fail
%% together). Following the defined overlay structure, this can be done through a
%% navigating scheme where at least one bridge is encountered. Monitoring this
%% replica can be done easily by having a \emph{watcher} in the same local group as
%% the replica.

\subsection{Reliability Mechanisms}
%These mechanisms should ensure  reliability and high availability of the
%\discovery system despite the scale and dynamicity of the underlying physical
%infrastructure.   
In a LUC, failures will be much more frequent than in actual UC
platforms. Furthermore, since resources could be highly distributed,
the expected mean time to repair failed equipments might be much
larger than in other UC platforms. For all these reasons, a set of
dedicated mechanisms should be designed in order to provide fully
transparent failure management with minimum downtime. 

%% It includes
%% mechanisms to ensure the high availability of the \discovery system
%% itself and mechanisms to allow executing VEs reliably.

Ensuring the high availability of the \discovery system requires being able to
autonomously restart any service by relocating it on a healthy agent each time
it is mandatory. To avoid losing or corrupting important information regarding
the state of the system, a Cassandra-like system~\cite{lakshman:2010} is
required to provide a reliable and highly available back-end for stateful
services.

Regarding VEs reliability, a first level of fault tolerance can be
provided by leveraging VMs snapshotting capabilities. Periodical
snapshots will allow restarting the VE from its last snapshot in the
event of a failure. Performing VM snapshotting in a large-scale,
heterogeneous, and widely spread environment is a challenging
task. However, we believe that adapting recently proposed
ideas~\cite{nicolae:2011} in this field would allow us to provide such
a feature.

Snapshotting is not enough for services that should be made highly
available, but a promising solution is to use VM
replication~\cite{Petrovic2012}. To implement VM replication in a WAN,
solutions to optimize synchronizations between
replicas~\cite{gerofi:2012,rajagopalan:2012} should be
investigated. Also, we think that a LUC has the major advantage over
other UC platforms, that it is tightly coupled with the network
infrastructure. As such, we can expect \emph{low} latencies between
nodes and so, to be able to provide strong consistency between
replicas while achieving acceptable response time for the replicated
services. 
%Note that the overlays presented in Section~\ref{ssec:p2p}
%for replicas localization and monitoring. Locating replicas based on a
%navigating scheme where at least one bridge is encountered, would be
%enough to ensure that they have a low probability to fail
%simultaneously.


Reliability techniques will of course make uses of the overlays
for resource localization and monitoring. 
Replicated VMs should be hosted on nodes that have a low
probability to fail simultaneously. Following the previously defined
overlay structure, this can be done through a navigating scheme where
at least one bridge is encountered. Monitoring a replica can then be
done by having a \emph{watcher} in the same local group as the
replica.

%% Another key use of this low-level overlay is proactive replication of VMs,
%% keeping in mind that two identical VMs should be placed in relatively distant
%% nodes, for fault-tolerance reasons (close nodes have a high probability to fail
%% together). Following the defined overlay structure, this can be done through a
%% navigating scheme where at least one bridge is encountered. Monitoring this
%% replica can be done easily by having a \emph{watcher} in the same local group as
%% the replica.




%
%% To this aim, the \discovery system should include, first,  mechanisms in charge of its own reliability. 
%% Such mechanisms are required to avoid losing or corrupting important information
%% regarding the state of the system.  Of course, handling all kinds of failures
%% and implementing a fully resilient operating system is a complex task. Hence,
%% we propose to consider in a first time a crash-stop failure model. In other words, the
%% \discovery system should be able to autonomously restart any service by
%% relocating it on an healthy agent each time it is mandatory.  This implies to
%% define a common design pattern, \textit{i.e.} a set of recommendations, that each
%% service should follow to ensure such characteristics. Besides, in order to
%% ensure that such a pattern may be applied to stateful services, the system can
%% require a Cassandra like system \cite{lakshman:2010} that provides reliable and highly available
%% storage for critical system states. Retrieving the information related to one
%% service need to rely on the kind of mechanisms described in Section~\ref{ssec:p2p}.

%% %
%% In addition to be robust enough, the \discovery system should ensure the
%% reliable execution of virtual environments. The first mechanisms consists in
%% using snapshotting capabilities delivering by virtualization technologies.
%% Concretely, each internal states of a VM should be periodically saved on a
%% persistent storage.  When a crash occurs on a VM, its associated VE can be
%% restarted from the latest consistent states, \textit{i.e.} all VMs of the VE will be
%% resume for their latest snapshot.  As for the other mechanisms, performing VM
%% snapshotting in a large-scale, heterogeneous and widely spread environment is a
%% challenging task. However, we believe that aforementioned mechanisms in charge
%% of the VM images as well as recent proposals \cite{nicolae:2011} might enable
%% to provide such a feature. 
%% Although VM snapshotting provides a first level of reliability, it is not
%% sufficient to ensure high availability of the VE. More advanced mechanisms must
%% be proposed.  Our idea is to include mechanisms based on primary-backup
%% replication techniques. 
%% The basic principle is to have one active replica of the VM (the primary)
%% sending state updates to the other replicas (the backups) periodically. If the
%% primary fails, one of the backup can resume the execution transparently for the
%% outside world. Furthermore since the entire VM is replicated, applications can
%% be run unmodified.  Solutions to replicate VMs inside a cluster have been
%% proposed. However efficiently replicating VMs over a WAN is a huge challenge.
%% Limiting the size of the backup updates\cite{rajagopalan:2012}, and
%% reducing the impact of the required synchronizations on the execution of the
%% primary \cite{gerofi:2012}  are research directions to be further
%% studied. A better understanding of the parts of a VM that really need to be
%% updated is required. It might require to trade transparency for performance by
%% allowing latency-sensitive applications to define which part of their state has
%% to be updated.

%% \ftodo[AL]{Integrate the P2P aspect into the reliability section}
%% \subsubsection*{Monitoring Example}

%% Another key use of this low-level overlay is proactive replication of VMs,
%% keeping in mind that two identical VMs should be placed in relatively distant
%% nodes, for fault-tolerance reasons (close nodes have a high probability to fail
%% together). Following the defined overlay structure, this can be done through a
%% navigating scheme where at least one bridge is encountered. Monitoring this
%% replica can be done easily by having a \emph{watcher} in the same local group as
%% the replica.


\subsection{Security and Privacy Mechanisms}
%
%LUC OS should include a mechanism for the security of the lower layers and particularly multiple P2P overlays.
%% Previous research~\cite{sybilattacks} has shown that without trusted identities, it 
%% is not possible to protect a P2P overlay against sybil attacks. 
%Recent advances~\cite{Castro:2002:SRS:844128.844156} in DHT 
%security might enable to provide a secured overlay. 
%But they are not sufficient to ensure the good behavior of resources and users.
%LUC OS should include an authentication and certification mechanism that 
%evaluates the behavior of resources and VEs.
%% Finally, as the resources could be hosted in more or less secured locations and provided by different 
%% resource providers, this certification should also be used to rank the security of each resources.
%% \ftodo[AL->JRC]{Could you please put only the most relevant one}
%
%Moreover, LUC OS shoud integrate security decision and enforcement points (SDEPs).
%%  mechanisms at all locations and layers.
%To guarantee a security, 
%Sandhu~\cite{sandhu_towards_2010} propose a roadmap towards such security mechanisms for
%Cloud where the Cloud infrastructure provides security hooks and
%mechanisms. 
%% To realize this goal, they point out the need for
%% \emph{developing models, methodologies and architectures for
%%   decentralized dynamic management of security and assurance
%%   policies}. 
%Similarly than other mechanisms, 
%As LUC infrastructure is even more distributed, heterogeneous and 
%dynamic than Cloud, its SDEPs should be distributed and 
%decentralized and able to evolve according to security requirements 
%without central security decision points. Futhermore, 
%LUC OS should provide a protocol to ease the collaboration between the SDEPs.
%Indeed, when new resources join
%a LUC infrastructure, SDEPs from different locations 
%must collaborate to extend the LUC infrastructure while keeping it secured. This collaboration between 
%SDEPs will also append when a VE is spread over different 
%resources and/or migrate from one to another.
%
%LUC OS should allow users to express their security requirements on their VEs.
%The expression of these requirements itself is a complex task.
%% LUC OS must propose to the user a way to model their VEs and 
%% the security requirements on them. 
%To ease the expression of these 
%requirements, LUC OS must propose a domain specific security language 
%that defines high-level security requirements such as \cite{rouzaud_book13b,Bacon:2010:EEA:2023718.2023739} do 
%for Clouds. These security requirements will be enforced 
%by SDEPs during the whole life time of the VE.
%
%%%%
% 1./ Securiser les overlays
% 3.a./ Definir  un DSL pour les VEs
% 3.b/ Offrir des moyens d'integrer des SDEPs (i.e. moulinette qui maintient/execute les DSLs)
% 2./ Securiser l'acces aux operations de manipulation de l'infra (admin et user)

% 1
To be successful, \discovery needs to provide mechanisms and methods to construct trust relationships between resource providers.
Trust relationships are known to be complex to build~\cite{Miller:2010:TWT:1907636.1907726}. Providing strong authentication, assurance and certification mechanisms for 
providers and users is required but it is definitely not enough. Trust covers socio-economic aspects that must be addressed but are out of the scope of this chapter.
The challenge is to provide a trusted \discovery base.

% 2 
As overlays are fundamentals for all \discovery mechanisms, another challenge is to ensure
that they are not compromised. Recent advances~\cite{Castro:2002:SRS:844128.844156} might enable to tackle such concerns.

%3.A ? 
The third challenge will consist in (i)~providing end-users with a way to define their own security and privacy policies and (ii)~ensuring that these policies are enforced. 
The expression of these policies itself is a complex task, since it requires
to improve the current trade-off between security (and privacy) and usability.
To ease the expression of these policies, we are currently designing a domain specific language
to define high-level security and privacy requirements~\cite{rouzaud_book13b,alefray:hpdc:2013}. 
These policies will be enforced in a decentralized manner,
by distributed security and privacy decision and enforcement points~(SPDEPs) during the lifetime of the VEs.
Implementing such SPDEP mechanisms in a distributed fashion will require to
conduct specific research, since currently there are only prospective proposals for
classic UC infrastructures~\cite{Bacon:2010:EEA:2023718.2023739,sandhu_towards_2010}. 
Therefore, we need to investigate whether such proposals can be adapted to the LUC
infrastructure by leveraging appropriate overlays.
%%  

% Finally, as traditional distributed systems, \discovery should ensure the good behavior of resources and users.
% To this aim, authentication and certification mechanisms should be provided. 

\subsection{Toward a First Proof of Concept}

The first prototype is under heavy development. It aims at delivering a
simple mock-up for integration/collaboration purposes.  Following the
coarse-grained architecture described in the previous sections, we have
started to identify all the components participating in the system, their
relationships, as well as the resulting interfaces. 
Conducting such a work now is mandatory to move towards a more complete as well
as more complex system. 

To ensure a scalable and reliable design, we  chose to rely on the use
of high-level programming abstractions; more precisely, we are using distributed
complex event programming~\cite{janiesch:2011} in association with the actors
model~\cite{agha:1986}. This enables us to easily switch between a push and a pull
oriented approach depending on our needs. 
%To make the integration of the mechanisms easier, we propose to follow a Service
%Oriented Architecture \cite{valipour:iccsit09}.

Our preliminary studies showed that a common building block is mandatory to
handle resiliency concerns in all components. Concretely, it corresponds to a
mechanism in charge of throwing notifications that are triggered by the low
level network overlay each time a node joins or leaves it.  Such a mechanism
makes the design and the development of higher building blocks easier as they do
not have to provide specific portions of code to monitor infrastructure
changes. 

This building block has been designed around the \emph{Peer Actor} concept (see Figure~\ref{fig:supervisor} and
Figure~\ref{fig:peeractor}).
 The \emph{Peer Actor} serves as an interface between higher services
and the communication layer. It provides methods that enable to define the behaviors of a
service when a resource joins or leaves a particular peer-to-peer overlay as well as
when neighbors change.
Considering that several overlays may co-exist in the \discovery system, 
the association between a \emph{Peer Actor} and its \emph{Overlay Actor} is
done at runtime and can be changed on the fly if need be. However, it is noteworthy that 
each \emph{Peer Actor} takes part to one and only one overlay at the same time.  
%
In addition to the \emph{Overlay Actor}, a \emph{Peer Actor}  is composed of a
\emph{Notification Actor}  that processes events and notifies registered actors.
%
As illustrated in Figure~\ref{fig:peeractor}, a service can use more than one \emph{Peer Actor} (and reciprocally). 
Mutualizing a \emph{Peer Actor} enables for instance to reduce the network overhead implied by the maintenance of the overlays. 
In the example, the first service relies on a \emph{Peer Actor} implementing a Chord
overlay~\cite{stioca:ton03}, while the second service uses an additional \emph{Peer Actor} implementing a CAN structure~\cite{ratnasamy:sigcomm01}.
 
\begin{figure}
  \begin{minipage}[c]{.35\linewidth}
	\vspace*{.5cm}
   \hspace*{-0.5cm}
      	\centering \includegraphics[width=2.8cm]{./FIGS/PeerActor.pdf}
   \hspace*{0.5cm}
	\vspace*{.5cm}
		\caption{The \emph{Peer Actor} Model. The \emph{Supervisor Actor} monitors all the actors it encapsulates while the \emph{Peer Actor} acts as an interface between the services and the overlay.}
\label{fig:supervisor}
   \end{minipage}
\hspace*{0.6cm}
   \begin{minipage}[c]{.55\linewidth}
   	\centering \includegraphics[width=4cm]{./FIGS/PeerActorServices.pdf}
	\vspace*{-0.25cm}
		\caption{A \emph{Peer Actor} Instantiation. The first service relies on a \emph{Peer Actor} implementing a Chord
overlay while the second service uses an additional \emph{PeerActor} implementing a CAN structure.}
		\label{fig:peeractor} 
  \end{minipage} \hfill
	\vspace*{-0.4cm}
\end{figure}

By such a mean, higher-level services can take the advantage of the advanced
communication layers without dealing with the burden of managing the different
overlays. As an example, when a node disappears, all services that
have been registered as dependent on such an event are notified. 
Service actors can thus react accordingly to the behavior that has been specified. 
%
%Finally, to ensure the reliable execution of the different actors, each one is supervised by a parent entitled 
%the {Supervisor Actor} (see Figure \ref{fig:SupervisorActor}).
%
%
% we consider that at this scale, failures are the norm
%rather than the exception, so we decided that each actor will be monitored
%by a \emph{Supervisor actor}. \discovery services are under the supervision of the
%\
%
%: an actor may crash for a variety of reasons
% we decided that each of the system's actors will be
%supervised by a parent actor called \emph{Supervisor actor} (figure 
%\ref{fig:SupervisorActor}): an actor may crash for a variety of reasons
%(network disruption, byzantine failure, etc.) and it is normal to consider that
%different kind of failures can lead to different reactions of the system. These 
%reactions will be decided by the \emph{Supervisor actor}: reboot of the actor,
%escalation of the error, etc.

Regarding the design and the implementation of the \discovery system, each
service is executed inside its own actor and communicates by
exchanging messages with the other ones. This ensures that each
service is isolated from the others: When a service crashes and needs to be
restarted, the execution of other services is not affected. 
% Besides, each service will be supervised by a global actor that encapsulates all
% services, ie. the \discovery agent. Each time a service fails, the \discovery
% agent catches and analyzes the failure so that it can decide which operations
% to perform. 
%
As previously mentioned, we consider that at the LUC infrastructure scale, failures are the norm
rather than the exception; hence we decided that each actor would be monitored
by a \emph{Supervisor Actor} (see Figure~\ref{fig:supervisor}). \discovery services are under the supervision of the \discovery agent: This design allows to precisely define a strategy
that will be executed in case of service failures. This will be the way to
introduce self-healing and self-organizing properties to the \discovery system.

This building block has been fully implemented\footnote{Code is available at:
\href{https://github.com/BeyondTheClouds}{\url{https://github.com/BeyondTheClouds}}} by 
leveraging the Akka/Scala framework~\cite{akka:www}.

As a proof of concept, we are implementing a first high level service in charge
of dynamically scheduling VMs across a LUC infrastructure by leveraging the
DVMS~\cite{quesnel:2012} proposal (see Section~\ref{ssec:vem}). The low-level
overlay that is being currently implemented is a robust ring based on the  Chord  algorithm
combined with the Vivaldi positioning system: It enables services to select nodes that have
low latency, so that collaboration will be more efficient. 

% When a node cannot guarantee the QoS for its
%hosted VMs, it starts an iterative scheduling procedure (ISP) by querying its
%neighbor to find a better placement. If the request cannot be satisfied by the
%neighbor (i.e., there is no viable placement to ensure the QoS of all VMs
%hosted on both reserved nodes), the request is forwarded to the following free
%node that takes part in that ISP.  The `absorption' of a new node is repeated
%until the ISP succeeds.  This approach allows DVMS to consider only a minimal
%number of nodes, thus decreasing the scheduling time without requiring a
%central coordinator.  Moreover, it allows several ISPs to occur independently
%at the same moment throughout the infrastructure.  To prevent deadlocks that
%may occur when all nodes are reserved by active ISPs, a distributed deadlock
%prevention mechanism has been designed. It enables DVMS to merge pairs of
%partitions.  To sum things up, the DVMS proposal has been designed to be fully
%distributed, non-predictive and event-driven by using partial views of the
%system.  Such a mechanism perfectly fits the requirements of a LUC OS.  The
%missing part was to extend DVMS in order to take into account resiliency
%aspects. This is going to be solved as we are implementing the DVMS algorithm
%with the \emph{Peer Actor}. 
%

%To validate the behavior, the performance as well as the reliability of our
%POC, we rely first on the Simgrid~\cite{Casanova:2008:SGF:1397760.1398183}
%toolkit. Simgrid has been recently extended to integrate virtualization abstractions and accurate migration models. 
%Simulations enable us to analyze particular situations and get several metrics
%that cannot be easily monitored  on a real platform.
%Second, results obtained from simulations are then compared to real experiments on the Grid'5000 platform. 
%Grid'5000 provides a testbed supporting experiments on various types of
%distributed systems (high-performance computing, grids, peer-to-peer systems,
%Cloud Computing, and others), on all layers of the software stack. The core
%testbed currently comprises 10 sites geographically spread across France. 
%For the Discovery purpose, we developed a set of scripts that enables to deploy in a \emph{one-click} fashion a large number of VMs throughout 
%the whole infrastructure\cite{flauncher}. 
%
To validate the behavior, the performance as well as the reliability of our
proof of concept~(POC), we are performing several experiments on the Grid'5000 testbed\footnote{\url{https://www.grid5000.fr}}
that comprises hundreds of nodes distributed on 10 computing sites that are geographically spread across France. 
To make experiments with \discovery easier, we developed a set of scripts that can deploy thousands of VMs throughout 
the whole infrastructure in a \emph{one-click} fashion~\cite{flauncher}.  
By deploying our POC on each node and by
leveraging the VM deployment scripts, we can evaluate real scenario by injecting specific workloads in the different VMs. 
The validation of this first POC is almost completed. 
The resulting system will be the first to provide reactive,
reliable and scalable
reconfiguration mechanisms of virtual machines in a fully distributed and
autonomous way. This new result will pave the way for a complete proposal of
the \discovery system. 




%%%
\section{Future Work/Opportunities\label{sec:future}}

%\subsection{How Evaluating LUC proposals}
%
%We first need to evaluate the quality of our LUC design in terms of performance and fault
%tolerance. For such a large system over a wide infrastructure, getting an analytical model
%is indeed a tough task. To make sure that the overall systems is efficient and reliable,
%simulation and experimentation over an actual platform are the best way to evaluate it. We
%will first rely on simulation to validate independent parts of the overall
%system. Simgrid~\cite{Casanova:2008:SGF:1397760.1398183} will be our platform
%of choice for this evaluation. We are currently adding virtualization
%simulation capabilities to the simulator. Our second evaluation platform will
%be Grid'5000. It provides a testbed supporting experiments on various types of
%distributed systems (high-performance computing, grids, peer-to-peer systems,
%cloud computing, and others), on all layers of the software stack. The core
%testbed currently comprises 10 sites. 
%%Grid'5000 is composed of almost 8000 CPU cores, with various generations of technology, CPUs from one to 12 cores,
%%Myrinet, Infiniband, and 2 GPU clusters). 
%A dedicated 10 Gbps backbone network
%is provided by RENATER. 
%%In order to prevent Grid'5000 machines from being the
%%source of a distributed denial of service, connections from Grid'5000 to the
%%Internet are strictly limited to a list of whitelisted data and software
%%sources, updated on demand. Users are allowed to install their own software
%%stack and run their experiment on a dedicated hardware. Grid'5000 is indeed a
%%Hardware as a Service (HaaS) platform. The communications between clusters can
%%also use virtual networks.
%While the use of RENATER in the Grid'5000
%project was ``only'' to provide users with a dedicated network with several
%functionalities such as XXX, in the DISCOVERY project and in collaboration with RENATER
%engineers, we will deploy a specific testbed on top of their network servers.

\subsection{Geo-Diversification as a Key Element}
The Cloud Computing paradigm is changing the way applications are designed.  In
order to benefit from the elasticity capability of Cloud systems, applications
integrate or leverage mechanisms to provision resources, \textit{i.e.} starting or
stopping VMs, according to their fluctuating needs.
The ConPaaS system \cite{pierre:2012} is one of the promising systems for elastic Cloud
applications. At the same time, a few projects have started investigating
distributed/collaborative way of hosting famous applications such as Wikipedia
or Facebook-like  systems by leveraging volunteer computing techniques. 
However, considering that resources provided by end-users were not reliable enough, only few contributions 
have been done yet. 
%
By providing a system that will enable to operate widely spread but more
reliable resources closer to the end-users, the LUC proposal may strongly
benefit to this research area.
Investigating the benefit of locality provisioning (i.e. combining elasticity and distributed/collaborative
hosting) is a promising direction for all web services that are embarrassingly distributed
\cite{church:2008}.  Image sharing system such as Google Picasa  or Flickr  are
examples of applications where leveraging locality will enable to limit network exchanges:
Users could upload their images on a peer close to them and images would be
transferred to other locations only when required (pulling vs. pushing
model).

LUC infrastructures will allow envisioning a wider range of services that may
answer specific SMEs requests such as data archiving or backup solutions while
significantly reducing the network overhead as well as legal concerns. Moreover, 
it will make the deployment of UC services easier by relieving developers of the burden of dealing with
multi-cloud vendors.
Of course, this will require software engineering and
middleware advances to easily take advantage of locality but proposing LUC OS
solutions such as the  \discovery project is the mandatory step before
investigating new APIs enabling applications to directly interact with the LUC OS internals. 
%
%A particular issue is to design adequate abstractions that LUC OS has to provide with
%respect to application description.  Indeed, specific interactions have to be set up so
%that applications can leverage LUC OS capabilities, for interactive and batch
%applications. It will require to understand what can/has to be handle at the LUC OS and
%what is the responsibility of the application (or its runtime). The risk is to have
%independent adaption loops at several levels of the software stack. Although a platform such
%as DISCOVERY is designed to provide IaaS offers, we should start to investigate how PaaS
%and SaaS solutions can benefit from the LUC properties directly in their internals by
%delivering the right API enabling applications to directly interact with the DISCOVERY OS.
%
\subsection{Energy, a Primary Concern for Modern Societies}

The energy footprint of current UC infrastructures and more generally of the
whole Internet is a major concern for the society.  By its design and the way
to operate it, a LUC infrastructure will have a smaller impact.
 Moreover, the LUC proposal is an interesting way to
deploy the data furnaces proposal \cite{liu:hotcloud11}.  Concretely, following
the smart city recommendations (i.e. delivering efficient as well as
sustainable ICT services), the construction of new districts in metropolises
may take the advantage of each LUC/Network PoP in order to heat buildings while
operating UC resources remotely thanks to a LUC OS. Finally, this idea might
be extended by taking into account recent results about passive data centers,
such as solar-powered
micro-data centers\footnote{\href{http://parasol.cs.rutgers.edu}{\url{http://parasol.cs.rutgers.edu}}}.
The idea behind passive computing facilities is to limit has much as possible
the energy footprint of major hubs and DSLAMS by taking advantage of renewable
energies to power them and by using the heat they product as a source of
energy. Combining such ideas with the LUC approach would allow reaching an
unprecedented level of energy efficiency for UC platforms.


%%%
\section{Conclusion\label{sec:conclusion}}

Distributing the management of Clouds is a solution to favor the adoption of the distributed cloud model. In this paper, we presented our view of how
such distribution can be achieved by presenting the premises of the LUC Operating System.
%We highlighted that it has  however a design cost
%and it  should be  developed over  mature and efficient  solutions:
We chose to develop it by leveraging the OpenStack solution. This choice
presents two advantages. It minimizes the development efforts and maximizes the
chance of being reused by a large community. As a proof-of-concept we presented
a revised version of the Nova service that uses a NoSQL backend. We discussed
few experiments validating the correct behavior and showed promising performance
over 8 clusters.
 
Our ongoing activities focus on two aspects. First, we expect
to finalize the same modifications on the Glance image service soon and start to
investigate also whether such a DB replacement can be achieved for Neutron. We
highlight that we chose to concentrate our effort on Glance as it is a key
element to operate an OpenStack IaaS platform. Indeed, while Neutron is becoming
more and more important, the historical network mechanisms integrated in Nova
are still available and intensively used. The second activity studies how it can
be possible to restrain the visibility of some objects manipulated by the
different controllers that have been deployed throughout the LUC infrastructure:
%Indeed some Finally, having a wan-wide infrastructure can be source of
%networking overheads:
our POC manipulates objects that might be used by any instance of a service, no
matter where it is deployed.
%On the  other hand, some  objects may
%benefit from a  restrained visibility:
If a user has build an OpenStack project (tenant) that is based on few sites,
appart from data-replication, there is no need for storing objects related to
this project on external sites. Restraining the storage of such objects
according to visibility rules would save network bandwidth and reduce
overheads.% in addition to enabling users to settle policies for applications
%such as privacy and efficient data-replication.

Although delivering an efficient distributed version of OpenStack is a
challenging task, we believe that addressing it is the key to go beyond
classical brokering/federated approaches and to promote a new generation of
cloud computing more sustainable and efficient. We are in touch with large
groups such as Orange Labs and are currently discussing with the OpenStack
foundation to propose the Rome/REDIS Library as an alternative to the
SQLAlchemy/MySQL couple. Most of the materials presented in this article such as
our prototype are available on the Discovery initiative website.

%Indeed,
%revising  OpenStack, in  order to  make  it natively  cooperative, would  enable
%Internet  Service Providers  and other  institutions  in charge  of operating  a
%network backbone  to build  an extreme-scale LUC  infrastructure with  a limited
%additional cost.

%% The  interest of important actors such as  Orange Labs that has
%% officially announced its  support to the initiative is an  excellent sign of the
%% importance of our action.       


%\begin{acknowledgement}
%If you want to include acknowledgments of assistance and the like at the end of an individual chapter please use the \verb|acknowledgement| environment -- it will automatically render Springer's preferred layout.
%\end{acknowledgement}
%

%% APPENDIX
%\section*{Appendix}
%\addcontentsline{toc}{section}{Appendix}
%
%
%When placed at the end of a chapter or contribution (as opposed to at the end of the book), the numbering of tables, figures, and equations in the appendix section continues on from that in the main text. Hence please \textit{do not} use the \verb|appendix| command when writing an appendix at the end of your chapter or contribution. If there is only one the appendix is designated ``Appendix'', or ``Appendix 1'', or ``Appendix 2'', etc. if there is more than one.

%\input{referenc}
\bibliographystyle{abbrv}
\vspace*{-.5cm}
\bibliography{main}
\end{document}
