\begin{abstract}
  
The use of computing ressources delivered by cloud providers has become
very common. These providers leverage on big data-centers containing tens of 
thousands of servers. However concentrating ressources in a few data-centers 
leads to critical situations in term of performance, reliability and data 
confidentiality. Leveraging the micro DC concept \cite{greenberg:2008}, we 
propose to address such concerns by operating cloud resources in a fully
decentralized manner.\\
Although a reference architecture describing fundamental services constituting 
an IaaS manager has been proposed in \cite{moreno2012iaas}, a detailed overview
of the mechanisms that are needed to build a massively distributed cloud is still missing.\\
In this paper, we depict a detailed map of the LUC-OS, a totally decentralized
IaaS manager that will massively spread the production of computing ressources.
As it can be considered a standard, we start from OpenStack and give a 
detailed view of mechanisms that needs to be added or revisited to work 
efficiently at a massive scale. Many of today's most advanced components have 
been developed separately from other components, making them working smoothly at
a massive scale is the focus of our work.

\keywords{Cloud computing, IaaS Architecture, locality, OpenStack,
 peer to peer, network overlay, vivaldi}
\end{abstract}


