\section{Reference architecture for IaaS clouds}
\label{sec:moreno}

% - Cloud Computing -> like electricity market.
% - Big fishes have emerged -> their APIs have become defacto standard.
% - emerging market -> yet no ISO norm, but standardization is coming.
% - Despite recent efforts, "data lock-in" still exists.
% - This lack of standardization is a drag on Cloud computing adoption.
Cloud computing is a model where companies consume computing resources produced 
by external cloud providers. This model has lead to a structure similar to the 
electricity market: in the same way electrical providers have build huge 
infrastructures for generating and transporting energy, Cloud providers now 
leverage data-centers that can host more than tens of thousands of servers. As 
this market is dominated by few actors (Amazon, Google, Microsoft, Rackspace, 
...), de facto standards have emerged. As this an emerging industry, 
normalization is still at an early stage: several ISO norms will been soon 
released and many consortiums have been created to promote the use of standards 
by Cloud Computing actors. Furthermore, despite recent efforts made to avoid 
"data lock-in" migration from one Cloud provider to another one is still 
difficult to achieve. All these factors are a drag on the adoption of Cloud 
computing.

% - To enable companies to manage their data in the way they want 
%   -> private clouds.
% - OpenSource IaaS manager have been developed (OpenStack, Open Nebula
%   ,...) to host private clouds.
% - Some studies compared them:
%    * They manipulate the same concepts (VMs, Networks, Storages, ...).
%    * They have architectural differences (CloudStack -> monolithic, OpenStack 
%      -> components, Eucalyptus -> AWS).
As an alternative to keep the control on their data, organizations can leverage
their own internally hosted Cloud infrastructure : private clouds enable the 
utilization of Cloud computing without depending to a Cloud provider. For this
purpose, many open source project for building an IaaS manager have emerged, 
thus enabling every organization to host its own private cloud. Several of these 
projects have become very popular: their communities have reached a critical 
size and many of the are used by large actors (Rackspace and IBM use OpenStack, 
CloudStack is used by cloud.com). Recent studies like \cite{peng:2009} have 
compared state of the art open-source IaaS manager and it appears that while 
they almost manipulate same concepts:
\begin{itemize} 
  \item Virtual machines are the smallest unit for delivering computing 
  resource.
  \item Networking and advanced networking (virtual networks, software defined 
  network, ...).
  \item Distribution and virtualisation of storage devices.
  \item Security tools.
\end{itemize} 
, there also exists many architectural differences such as:
\begin{itemize} 
  \item CloudStack's development has been oriented towards a monolithic 
  architecture.
  \item OpenStack's architecture is composed of specialized managers.
  \item Eucalyptus follows the Amazon AWS services' architecture linked via a
  message bus.
\end{itemize}, thus leading to an heterogeneity that is a drag to the adoption 
of Cloud Computing.

% - Cloud heterogeneity -> No consensus on what a cloud do (OpenStack -> private
%   clouds, Eucalyptus -> AWS like platform, ...) and what interfaces it should
%   follow.
% - A recent study depicts a reference architecture for IaaS manager: the IaaS
%   manager is subdivided on several specific managers.
% - Description of each main managers. Short description of secondary managers.
%    for each components -> challenge of the components.
This architectural heterogeneity can be partially explained by the lack of 
consensus of what is the goald of an IaaS manager is: while Eucalyptus focuses
on building an infrastructure that shares the same API as Amazon AWS, projects 
like OpenStack have targeted the delivering of a complete IaaS that can operate
large cloud infrastructure by using its own tools. To cope with these 
significant differences, Moreno et al. have proposed a reference architecture 
\cite{moreno2012iaas} for IaaS clouds. This architecture covers services that 
are needed for building a Cloud OS : each aspect of the cloud is supervised by a 
specific manager. The following listing gives a description of principal
services that authors expect in a Cloud OS:
\begin{description}
	
	\item [Virtual machines manager] is the most important part of the Cloud OS:
	it manages virtual machines' cycle of life (deployment, migration, 
	suspension, resume and shut down). It is also responsible for maintaining
	a good quality of service, especially for large scale infrastructure.

	\item [Scheduler] decides which server will run a newly created virtual 
	machine by taking into account several parameters (static scheduling). To 
	preserve a good quality of service, overloaded servers may migrate virtual
	machines on underloaded servers (dynamic scheduling).

	\item [Image manager] manages virtual machines' images. As users have their
	own images with specific systems or configurations, this manager must be 
	able to handle a large number of images in a distributed manner.

	\item [Network manager] provides connectivity to the infrastructure: virtual
	networks for virtual machines, external access for users. The current trends
	is to the virtualisation of networking devices through the use of Software
	Defined Networks (SDN). This manager must enable flexible networking 
	deployment without losing performance (bandwidth and latency).

	\item [Storage manager] provides persistent storage facilities to virtual 
	machines. As physical single storage device might be a source of failure,
	factors like distribution and redundancy must be anticipated.

	\item [Administrative tools] provides tools and user interfaces to perform
	administration tasks. This is the components that enables users acces to
	the Cloud OS internals.

	\item [Information manager, Accounting/auditing ] collects raw monitoring 
	data of the infrastructure and produce value-added from the raw data 
	respectively. As authors mentionned it, both are very important for billing
	process.

\end{description}

The other managers, not included in the preceding listing, are either taking 
place at different level in the Cloud Computing hierarchy (\emph{Service 
manager} is likely to be at PaaS level) or either too advanced for a first Cloud
OS prototype (\emph{Cloud Interfaces} exposes a cloud API to external users; 
\emph{Federation manager} enables the access to remote Clouds.) or is of little 
scientific interest for the Cloud OS (\emph{Information manager} gather VMs 
state information, as many tools meet this need, the Cloud OS will simply 
reuse them).

% - We want to develop a Cloud OS.
% - Develop from scratch: impossible (herculean work) -> maximize the reuse of
%   existing components.
% - We propose to leverage the reference architecture proposed by Moreno.
% - In order to develop a prototype, we focus of vital components.
We target the development of the LUC-OS : a fully distributed Cloud OS that can 
provides essential features proposed by current opens source IaaS managers. As
developing a complete Cloud OS is an herculean work, trying to maximize the 
reuse of existing works is a way to minimize both design and implementation 
efforts. Indeed, our first objective is to leverage the best practices 
established by the analysis of today's IaaS managers. We think that leveraging
this reference architecture is the most obvious way to cope with this first
objective.
