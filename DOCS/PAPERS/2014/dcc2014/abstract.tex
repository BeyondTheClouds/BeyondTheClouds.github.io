\begin{abstract}
  
The use of computing ressources delivered by cloud providers has become
very common. These providers leverage on big data-centers containing tens of 
thousands of servers. Concentrating ressources in data-centers is often the root
of single point of failure (SPOF) and data confidentiality issues. We propose 
to adress these problems by taking into account locality properties and
revisiting existing cloud mechanisms.
\\
Efforts to develop a common API for opensource IaaS managers seems 
unsuccessful: while many of them provide specific features, only a few of them 
schedules ressources dynamically. Although a reference architecture describing 
typical services consituting a typical cloud has been proposed in 
\cite{moreno2012iaas}, a detailed overview of the mechanisms that are needed to
build a massively distributed clouds is still missing.
\\
In this paper we propose an architecture for a massively distributed cloud that
takes into account locality properties. As it can be considered a standard, we 
start from OpenStack and give a detailed view of mechanisms that needs to be
added or revisited to work efficiently at a massive scale. Many of today's most
advanced components have been developed separately from other components, making
them working smoothly at a massive scale is the focus of our work.

\keywords{Cloud computing, IaaS Architecture, locality, OpenStack,
 peer to peer, network overlay, vivaldi}
\end{abstract}
