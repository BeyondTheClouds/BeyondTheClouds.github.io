\section{Revisiting existing cloud mechanisms}
\label{sec:integration}

\subsection{OpenStack: a toolkit for building clouds}

% \begin{itemize}
	
% 	\item Opensource project considered as a standard in cloud infrastrcutre.

% 	\item Composed of several "shared nothing architecture" services (nova, 
% 	swift, Quantum, Glance, ...).

% 	\item Uses an AMQP (Advanced Message Queuing Protocol) for inter-components 
% 	communication. => additional components can easily plug on an existing
% 	infrastructure.

% 	\item Drawack: OpenStack does not perform dynamic scheduling of VMs.

% \end{itemize}

OpenStack is a very popular project that aim at building an opensource IaaS
manager. A lot of Cloud computing actors (Rackspace, VMware, Cisco, ...) are
contributing to this project, thus providing new features at a rapide pace.

A cloud deployed with OpenStack is composed of several services (nova, swift,
Quantum, Glance, ...) following the share "shared nothing architecure", meaning 
that each service is independent and thus share no state.

Inter-services collaboration is performed by exchanging message through an AMQP 
(Advanced Messsage Queuing Protocol) based bus : each service has its own queue,
and they collaborate by sending message to remote service's queue. This offers
the advantage of enabling easy plugging of additional components : if one wants
to replace one default service by a custom service, naively consuming messages
destinated to the replaced service and , in turn producing expected message is
enough.

As OpenStack is becoming a standard and since it provides working mechansims, it
can be considerated as a cloudkit of choice for the LUC-OS. However OpenStack is
not totally in sync with its requirement: OpenStack performs static resource
scheduling: when a VM is created, nova-scheduler decides the server where it
should be started, without rescheduling it in case of resource overloading. That
is why we discuss the integration of DVMS as an OpenStack's service in section 
\ref{sec:future_work}.

\subsection{Revisiting components that are suitable for a LUC-OS (Related work)}
\label{sub:sec:revisiting_openstack}

% \begin{itemize}

% 	\item As we do not want to redevelop every services "from scratch", we will
% 	maximize the reuse of existing	mechanisms.

% 	\item In section \ref{sec:architecture} we defined a list of services provided 
% 	by the LUC-OS. Most of them will partially/completly leverage existing 
% 	mechanisms:

% 		\begin{description}

% 			\item [Compute manager] : 
% 			\begin{itemize}
% 				\item Nova is the service that is responsible for controlling 
% 				every services of the infrastucture. 

% 				\item It leverages nova-scheduler, which performs static 
% 				scheduling: nova-scheduler.

% 				\item We plan to use a dynamic scheduler (DVMS) instead.
% 			\end{itemize}

% 			\item [Administrative manager] : 
% 			\begin{itemize}
% 				\item KeyStone is the service used for managing identity and
% 				authentication.

% 				\item Horizon is the service that serves a web interface, 
% 				allowing users to interact with OpenStack.

% 				\item As we will use a distributed hash table (DHT) to save
% 				persistent states, we plan reuse Keystone and Horizon, with
% 				minor modification to leverage our DHT. 
% 			\end{itemize}

% 			\item [Storage manager] : 
% 			\begin{itemize}
% 				\item Swift is a distributed key/value service with no SPOF.
				
% 				\item Glance is a service for storing image that leverages 
% 				Swift. It is possible to directly store images in Swift.

% 				\item To limit network impact ("Boot problem"), solution based 
% 				on bittorrent have been developped (VMTorrent, 
% 				glance-bittorrent, ...).

% 				\item For a first prototype we propose to leverage Glance and
% 				Swift. If this solution become no more suitable for a massive
% 				infrastucture, we propose to integrate VMTorrent in Glance.

% 			\end{itemize}

% 			\item [Network manager] :
% 			\begin{itemize}
				
% 				\item Neutron is the service used to create VLAN.

% 				\item It supports plugins: it is possible to use Open vSwitch.
% 				This is what we propose for a first version of the system.

% 				\item For next versions, Mininet \cite{lantz:2010} or Vin can be
% 				considered as good candidates.
% 			\end{itemize}

% 		\end{description}


The objective behind the LUC-OS in to provide an IaaS manager that can work at 
massive scale by leveraging a locality aware overlay network (LBO). As 
developing a complete IaaS manager is an Herculean work, we think that is the
good way to reach our goal is to revisit existing mechanims, even if it means 
that some of these mechanisms may be completely replaced.

For the sake of clarity, the mechanisms we plan to revisit will be studied 
following the list of services depicted in section \ref{sub:sec:list_services} :
for each service we indicate what are the suitable mechanisms that will be 
revisited, and what we propose to change.

\begin{description}

	\item [Compute manager] : Nova is an OpenStack service that is 
	responsible for coordinating and controlling the work of other services.
	As it is the angular stone of OpenStack, we plan to directly plug the 
	LUC-OS on this service. Nova service contains a sub-service that is
	responsible for scheduling virtual machines: nova-scheduler. It performs
	static scheduling : it means that the scheduler will decide on which
	service the VM will run once and for all. In section 
	\ref{sub:sec:integration_dvms} we discuss the replacement of 
	nova-scheduler by DVMS.


	\item [Administrative manager] : KeyStone and Horizon are two services
	that are respectively responsible for managing identity/authentication
	and providing a web user interface to users that want to interact with
	OpenStack. The LUC-OS will store every piece of information in a DHT: we
	propose to leverage KeyStone and Horizon on the DHT. Furthermore the 
	LUC-OS and OpenStack does not exactly share the same semantics : for
	example the LUC-OS manipulates virtual environments which does not
	exactly exist in OpenStack. Revisiting Horizon service through minor
	modifications would enable the integration of this concept in OpenStack.

	\item [Storage manager] : Storage in OpenStack can be divided in two
	service: Swift manages objects (files) while Glance manages VM disk
	images. Glance's images can be stored in Swift or in any other storage
	service. Recent solutions like VMTorrent \cite{reich:2012} have been
	developed to limit network overhead, for example during simultaneous 
	loading of images at VMs startup. For a first prototype we propose to 
	leverage Glance and Swift, and if it becomes no more suitable for a
	massive infrastucture, we propose to integrate VMTorrent in Glance and 
	to use it.

	\item [Network manager] : Neutron is the service that manages networking
	in OpenStack: it is used to create virtual networks (VLANs) that
	interconnect virtual machines. Neutron supports plugins : it is possible
	to use a virtual multilayer switch like Open vSwitch \cite{pfaff:2009}.
	In the case this plugin does not meet our needs, we propose to 
	leverage software defined network (SDN) solution like Mininet 
	\cite{lantz:2010} or Vin.

\end{description}




