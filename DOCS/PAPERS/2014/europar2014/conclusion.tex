
Cloud Computing has entered our everyday life at a very high speed and huge scale. From classic high performance computing simulations to the management of huge amounts of data coming from mobile devices and sensors, its impact can no longer be minimized. While a lot of progress has already been made in Cloud technologies, there are several concerns that limit the complete adoption of the Cloud Computing paradigm.

In a previous report~\cite{lebre:hal-00854204}, we outlined that, in addition to these concerns, the current model of UC is limited by intrinsic issues. Instead of following the current trend by trying to cope with existing platforms and network interfaces, we proposed to take a different direction by promoting the design of a system that will be efficient and sustainable at the same time, putting knowledge and intelligence directly into the network backbone itself. The innovative approach we introduced will definitely tackle and go beyond Cloud Computing limitations. Our objective is to pave the way for a new generation of Utility Computing that better matches the Internet structure by means of advanced operating mechanisms. By offering the possibility to tightly couple UC servers and network backbones throughout distinct sites and operate them remotely, the LUC OS technology may lead to major changes in the design of UC infrastructures as well as in their environmental impact. The internal mechanisms of the LUC OS should be topology dependent and resources efficient. The natural distribution of the nodes through the different points of presence should be an advantage, which allows to process a request according to its scale: Local requests should be computed locally, while large computations should benefit from a large number of nodes.

The first step toward this highly distributed Cloud infrastructure taking into account locality and network distance is the scheduling of VMs taking into account locality. Thus is this paper, we presented our first building block of our distributed Cloud infrastructure that consists in relying on the Vivaldi protocol and P2P algorithms connected to DVMS, an efficient and flexible VMs scheduler. Our first experiments over Grid'5000 show that, connecting 4 differents sites and scheduling VMs over them, we can gain up to 66\% of inter-sites operations. It is worth noting that one experimental observation we had during this work is that the proposed overlay is actually able to maintain an overlay networks that reflects the underlying topology, and in particular to build a hierarchical overlay dynamically if the underlying topology is hierarchical.

Our future work will consist in ... 
