
Cloud Computing has entered our everyday life at a very high speed and huge scale. From
classic High Performance Computing simulations to the management of huge amounts of data
coming from mobile devices and sensors, its impact can no longer be minimized.
% While a lot of progress has already been made in Cloud
%technologies, there are several concerns that limit the complete adoption of
%the Cloud Computing paradigm.
While promoted for a long time, delivering Cloud Computing capabilities by leveraging
only few large-scale data centers does not enable to cope with the demand of Cloud resources
anymore, and a new model consisting in leveraging several micro/nano data centers
distributed WANwide is more and more investigated. The main challenge is thus to revisit
most of the mechanisms that are common to current IaaS management systems to
leverage more decentralized algorithms. Among the different contributions that have been
proposed, a large number have focused on the scheduling issue of the VMs to
achieve the scalability required but at the expense of the locality criteria. However,
manipulating VMs WANwide degrades significantly the performance as well as the quality of
the service of the whole system.

% 
%In a previous report~\cite{lebre:hal-00854204}, we outlined that, in addition
%to these concerns, the current model of UC is limited by intrinsic issues.
%Instead of following the current trend by trying to cope with existing
%platforms and network interfaces, we proposed to take a different direction by
%promoting the design of a system that will be efficient and sustainable at the
%same time, putting knowledge and intelligence directly into the network
%backbone itself. The innovative approach we introduced will definitely tackle
%and go beyond Cloud Computing limitations. Our objective is to pave the way for
%a new generation of Utility Computing that better matches the Internet
%structure by means of advanced operating mechanisms. By offering the
%possibility to tightly couple UC servers and network backbones throughout
%distinct sites and operate them remotely, the LUC OS technology may lead to
%major changes in the design of UC infrastructures as well as in their
%environmental impact. The internal mechanisms of the LUC OS should be topology
%dependent and resources efficient. The natural distribution of the nodes
%through the different points of presence should be an advantage, which allows
%to process a request according to its scale: Local requests should be computed
%locally, while large computations should benefit from a large number of nodes.

Hence, the first step toward such a highly distributed Cloud infrastructure is to take
into account this notion of locality between Cloud Computing resources. In this paper, we
showed how such locality criteria can be considered by delivering a new building block using
P2P algorithms and a Vivaldi overlay network connected to the DVMS proposal, an efficient and
flexible VM scheduler. Our first experiments over Grid'5000 showed that, connecting 4
different sites and scheduling VMs over them, we could gain up to 72\% of inter-site
operations. It is worth noting that one experimental observation we had during this work
was that the proposed overlay network was actually able to
reflect the underlying topology, and in particular to build a hierarchical overlay
dynamically if the underlying topology is hierarchical.

Our future work will consist in refining the decision model used in scheduling mechanisms
to enable them to consider the cost difference between intra-site and inter-site
migrations, thus promoting intra-site migrations in multi-site partitions. More generally,
the association between locality-based overlay networks and Peer Actors will become a building block
for revisiting every single service composing IaaS systems. It will enable to deliver a
new generation of Utility Computing as depicted by the Discovery
Initiative\footnote{\url{http://beyondtheclouds.github.io/}}.
