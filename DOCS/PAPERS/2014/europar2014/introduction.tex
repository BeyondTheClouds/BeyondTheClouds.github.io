\section{Introduction}
Introduced few years ago \cite{greenberg:sigcomm09}, the new trend to deliver
cloud computing resources, in particular Infrastructure as a Service solutions,
consists in leveraging several infrastructures  distributed world-wide. If
such distributed cloud computing platforms deliver undeniable advantages to address important
challenges such as reliability, latency or even in somehow jurisdiction
concerns, most mechanisms that were previously used to operate centralized IaaS
platforms must be revisited to offer the same level of transparency for the end-users. 
Keeping such an objective in mind, the use of P2P
paradigm is strongly investigated. 
This is particularly true for instance for scheduling algorithms in charge of assigning
VMs on top of PMs according to their effective needs (and reciprocally usages). 
Indeed and although major improvements have
been done, centralized approaches \cite{hermenier:2013} are not scalable enough
and hierarchical solutions \cite{feller:ccgrid2012} face important limitations
regarding the reactivity to take into account physical topology changes, an
important criteria in such widely distributed infrastructures. 
 
