\section{Introduction} Introduced few years ago \cite{greenberg:sigcomm09}, the
new trend to deliver cloud computing resources, in particular Infrastructure as
a Service solutions, consists in leveraging several infrastructures
distributed world-wide. If such distributed cloud computing platforms deliver
undeniable advantages to address important challenges such as reliability,
latency or even in somehow jurisdiction concerns, most mechanisms that were
previously used to operate centralized IaaS platforms must be revisited to
offer the same level of transparency for the end-users.  Keeping such an
objective in mind, the use of P2P paradigm is strongly investigated. This is
particularly true for instance for scheduling algorithms in charge of assigning
VMs on top of PMs according to their effective needs (and reciprocally usages).
Indeed and although major improvements have been done, centralized approaches
\cite{hermenier:2013} are neither scalable nor robust enough.  Hierarchical
solutions \cite{feller:ccgrid12} that can be seen as good candidates face
important limitations: First, finding an efficient partitioning of resources is
a tedious task as matching a hierarchical overlay on top of a distributed
infrastructure is often not logical.  Second, in addition to requiring complex
failover mechanisms to ensure leader/super peer crashed and network
disconnections, hierarchical structures have not been designed to react swiftly
to physical topology changes such as node apparitions/removals and network
performance degradations.  P2P algorithms enable to address both concerns, \ie
scalability as well as resiliency of infrastructures. However and although,
promising approaches have been proposed to address the scheduling problem of
VMs in a P2P fashion \cite{xxx,quesnel:2012,feller:cloudcom12}, current
proposals do not make differences between close nodes and far ones.
Considering that both the network latency and the bandwidth between peers are fundamentals since
they can have a strong impact on the reactivity criteria of the scheduling
algorithm (\ie the time to switch from one schedule to another one), \emph{locality} properties should be considered to favor 
efficient VM operations:~\eg  peers should collaborate first with their closest
neighborhoods in most cases from the same geographical site before contacting
the remote ones.  Moreover this notion of locality can fluctuate over time
according to the network bandwidth/latency that's why similarly to hierarchical
approaches, static partitions of resources is not appropriated. 
  
\AL[CT/MB]{Can we add few words here regarding Gossip and P2P mechanisms that can deal with locality aspects}. 

The contribution of this paper is a new building block that enables each peer
to collaborate with the closest ones in terms of \emph{locality}. Locality is
defined as a cost function of the latency/bandwidth tuple. 


Leveraging a vivaldi overlay, a shortest path algorithm  a
In this paper, we present a new building block built on a vivaldi overlay 
 
 
