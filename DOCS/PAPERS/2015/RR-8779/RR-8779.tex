\documentclass[a4paper,twoside]{article}

\usepackage{RR}
\usepackage{hyperref}
%\usepackage[frenchb]{babel}
%\usepackage[T1]{fontenc} % avec T1 comme option  d'encodage c'est ben mieux, surtout pour taper du français.
\usepackage[utf8]{inputenc}
\usepackage[table]{xcolor}
\usepackage{color}
\usepackage{graphicx}
\usepackage{amsmath, amsthm}
\usepackage{stmaryrd}
\usepackage{lscape}

\usepackage{url} \urlstyle{sf}%
\usepackage{graphicx}%
\usepackage{subfigure}
\usepackage{listings}%

\lstset{%
  basicstyle=\scriptsize,%
  numbers=left,
  %columns=fullflexible,%
  language=XML,%
  %frame=shadowbox,
    frame=lbtr,%
  frameround=tttt,%
  tabsize=2,
  breaklines=true
}%
\usepackage{tikz}
\usetikzlibrary{decorations.pathreplacing,positioning}
\usepackage{array}
\usepackage{xspace}

\newcommand{\ie}[0]{{\em i.e.},\xspace}
\newcommand{\vs}[0]{{\em vs.}\xspace}
\newcommand{\eg}[0]{{\em e.g.},\xspace}
\newcommand{\etal}[0]{{\em et al.}\xspace}
\newcommand{\wrt}[0]{{\em w.r.t.}\xspace}
\newcommand{\aka}[0]{{\em a.k.a.}\xspace}

\sloppy

%%% FORMAT DOCUMENT

\def\textfraction{0}
\def\floatpagefraction{1}
\def\topnumber{3}
\def\bottomnumber{3}
\def\totalnumber{3}
\def\topfraction{1}
\def\bottomfraction{1}

%%.
\usepackage{bold-extra,graphicx,latexsym,mathrsfs,subfigure,xspace}

\usepackage{color}
\usepackage{array}
\usepackage{longtable}
\usepackage{calc}
\usepackage{multirow}
\usepackage{hhline}
\usepackage{ifthen}

\usepackage{hyperref}

\newcolumntype{M}[1]{>{\raggedleft}m{#1}}
\newcommand{\discovery}{DISCOVERY\xspace}

%% CT %%
\newcommand{\ftodo}[2][\relax]  % \TODO[editor]{text} 
  {\ensuremath{{}^{\mbox{\tiny\bf #1}}}~\textbf{#2}}

\begin{document}

\RRNo{8779}
\RRdate{Sep 2015}

\RRprojet{Ascola, Asap, Avalon, Kerdata, Myriads}

% \author{Adrien L?bre\inst{1,3} \and Jonathan Pastor\inst{1,3} \and Marin Bertier \inst{2,3} \and Fr?d?ric Desprez\inst{3} \and Jonathan Rouzaud-Cornabas\inst{3} \and C?dric Tedeschi \inst{3,4}\and Paolo Anedda\inst{5} \and Gianluigi Zanetti\inst{5} 
% \and Ramon Nou\inst{6} \and Toni Cortes\inst{6} \and Etienne Riviere\inst{7} \and Thomas Ropars\inst{8}}
% \institute{LINA / Mines Nantes, France 
% \and IRISA / INSA de Rennes, France
% \and INRIA, France
% \and IRISA / Universit? de Rennes 1, France
% \and Center for Advanced Studies, Research and Development in Sardinia (CRS4), Italy
% \and Barcelona Supercomputing Center (BSC), Spain
% \and Universit? de Neuch?tel (UniNe), Switzerland
% \and Ecole Polytechnique F?d?rale de Lausanne (EPFL), Switzerland} 

%
% Title and Authors
%
\RRauthor{
A. Lebre\thanks[Inria]{Inria, France, Email: \url{FirstName.LastName@inria.fr}}\thanks[EMN]{Mines Nantes/LINA (UMR 6241), France.}%\inst{1} 
\and 
J. Pastor\thanksref{Inria}\thanksref{EMN} 
\and 
The Discovery Consortium\thanksref{Inria}  
}

\authorhead{A. Lebre et al.}

\RRtitle{L'initiative DISCOVERY - les infrastructures IaaS massivement distribuées comme solution aux principales limitations des plateformes de Cloud Computing actuelles}
\RRetitle{The DISCOVERY Initiative\\Overcoming Major Limitations of Traditional Server-Centric Clouds by Operating Massively Distributed IaaS Facilities}%\thanks{This report ....}}
\titlehead{The DISCOVERY Initiative}

%\RRnote {XXXX}

\RRkeyword{Locality-Based Utility Computing, 
Peer To Peer,
Self-*,
Sustainability,
Efficiency,
Future Internet.
}

\RRmotcle{Calcul utilitaire basé sur la localité, systèmes pair-à-pair, self-*, durabilité,
Internet du futur}

%
% Abstract
%
\RRabstract{

  {\small 

  Instead of the current trend consisting of building larger and
  larger data centers (DCs) in few strategic locations, the DISCOVERY
  initiative\footnote{\url{http://beyondtheclouds.github.io}}
 proposes to leverage any network point of presences (PoP, \ie a
  small or medium-sized network center) available through the
  Internet. The key idea is to demonstrate a widely distributed Cloud
  platform that can better match the geographical dispersal of users.
  This involves radical changes in the way resources are managed, but
  leveraging computing resources around the end-users will enable to
  deliver a new generation of highly efficient and sustainable Utility
  Computing (UC)
  platforms, thus providing a strong alternative to the actual Cloud
  model based on mega DCs (i.e. DCs composed of tens of thousands
  resources).

  Critical to the emergence of such distributed Cloud platforms is the
  availability of appropriate operating mechanisms. Although, some of
  protagonists of Cloud federations would argue that it might be
  possible to federate a significant number of micro-Clouds hosted on
  each PoP, we emphasize that federated approaches aim at delivering a
  brokering service in charge of interacting with several Cloud
  management systems, each of them being already deployed and operated
  independently by at least one administrator. In other words, current
  federated approaches do not target to operate, remotely, a
  significant amount of UC resources geographically distributed but
  only to use them. The main objective of DISCOVERY is to design,
  implement, demonstrate and promote a unified system in charge of
  turning a complex, extremely large-scale and widely distributed
  infrastructure into a collection of abstracted computing resources
  which is efficient, reliable, secure and friendly to operate and
  use.

  After presenting the DISCOVERY vision, we explain the different
  choices we made, in particular the choice of revising the OpenStack
  solution leveraging P2P mechanisms. We believe that such a strategy
  is promising considering the architecture complexity of such systems
  and the velocity of open-source initiatives.
}
}

\RRresume{

  {\small 

La tendance actuelle pour supporter la demande croissante d'informatique
utilitaire consiste à construire des centres de données de plus en plus grands,
dans un nombre limité de lieux stratégiques. Cette approche permet sans aucun
doute de satisfaire la demande actuelle tout en conservant une approche
centralisée de la gestion de ces ressources, mais elle reste loin de pouvoir
fournir des infrastructures répondant aux contraintes actuelles et futures en
termes d’efficacité, de juridiction ou encore de durabilité.  L’objectif de
l'initiative DISCOVERY\footnote{\url{http://beyondtheclouds.github.io}} est de
concevoir le \emph{LUC OS},  un système de gestion distribuée des ressources qui permettra de
tirer parti de n’importe quel n\oe ud réseau constituant la dorsale d’Internet
afin de fournir une nouvelle génération d’informatique utilitaire, plus apte à
prendre en compte la dispersion géographique des utilisateurs et leur demande
toujours croissante. 

Après avoir rappelé les objectifs de l'initiative DISCOVERY et expliqué
pourquoi les approches type fédération ne sont pas adaptées pour opérer une
infrastructure d'informatique utilitaire intégrée au réseau, nous présentons les
prémisses de notre système.  Nous expliquerons notamment pourquoi et comment
nous avons choisi de démarrer des travaux visant à revisiter la conception de
la solution Openstack. De notre point de vue, choisir d'appuyer nos travaux sur
cette solution est une stratégie judicieuse à la vue de la complexité des
systèmes de gestion des plateformes IaaS et de la vélocité des solutions
open-source. 
}
}

\URRennes
\makeRR

%%%

\subsection{Discovery Initiative}

\begin{itemize}

	\item Users manipulates virtual environment, which is the "gravity center of the IaaS".

	\item Virtual environment contains virtual machines.

\end{itemize}

\section{Designing a massively distributed Cloud}
\label{sec:design}


\subsection{Toolkit for IaaS}

% \begin{itemize}

% 	\item A toolkit is a building block that can be used for the construction of
% 	systems (generic definition of a software toolkit).

% 	\item The objective of a toolkit is to provide "state of the art" solutions
% 	to known problems. It enables the focus on "Top level" works.

% 	\item It provides a set of components, which once assembled constitute an 
% 	operational system.

% 	\item Recent studies of "state of art IaaS systems" (OpenStack, Cloudstack,
% 	OpenNebula, ...) showed that they were constructed over same concepts. It 
% 	enables the design of IaaS toolkit.

% 	\item The massively distributed IaaS toolkit will provides "state of the 
% 	arts" mechanism to solve both scalability and locality points.

% 	\item The toolkit will have to integrate well on existing systems: we 
% 	propose to leverage OpenStack project.

% \end{itemize}

A software toolkit is a set of software building blocks that includes state of 
the art mechanisms for known problems. A toolkit comes with an API (Application
Programming Interface) which is the specifications that ones must follow to
correctly use provided mechanisms. The goal of a toolkit is to enable developers
to focus on the creation of higher level mechanisms, thus speeding up the 
development time.

An IaaS toolkit should be delivered with a set of default high level mechanisms 
whose assembly results in an basic operational IaaS system. In the case where 
one of the default constituting mechanisms would not be sufficient, it should be
redeveloped by leveraging the toolkit's low level mechanisms.

Recent studies have showed that state of the art IaaS manager \cite{peng:2009}
were constructed over the same concepts. Furthermore a reference architecture 
for IaaS manager has been described in \cite{moreno2012iaas} enabling the design
of an IaaS toolkit. Besides the reference architecture, an IaaS toolkit should
provides some mechanisms that enables to solve the scalability problem.

To maximize the chance of being reused by a large community, an IaaS toolkit 
should enable an easy integration with one or several existing IaaS cloud
managers. In our case we have chosen to leverage the OpenStack project: as a 
result the mechanisms developped with the toolkit will integrate well with 
existing clouds that are based on OpenStack.



\subsection{Massively distributed cloud}

% \begin{itemize}

% 	\item A massively distributed cloud targets management of thousand of hosts 
% 	around a wide territory.

% 	\item This scale order is currently reached by file sharing systems like 
% 	bittorrent.

% 	\item At this scale, failure becomes the norm.

% 	\item Recent works propose to leverage on peer to peer overlay.

% 	\item Some peer to peer overlays can take advantage of locality. It enables
% 	to build systems that can take into account network bandwidth and latency.

% 	\item We propose to leverage on locality based peer to peer mechanisms to 
% 	reach an high scalability IaaS.

% \end{itemize}

Cloud providers concentrate the production of computing resources in 
data-centers that contains tens of thousand of servers, enabling IaaS mechanisms
to take advantage of fast network with extremely low latency. However this ever 
increasing data-centers size has become a problem, as many data-centers require 
dedicated electrical and cooling infrastructure. As an alternative to 
concentrating the production of computing resources, we propose to study a model
where this production is deconcentrated.

Leveraging the concept of micro data-centers proposed by \cite{greenberg:2008},
we suggest to build a cloud operating system that will run in a distributed
manner over a set a small data-centers geographically spread. This cloud 
operating system will have to reach high scalability criteria: managing 
thousands of servers used by hundreds of users. Popular peer to peer file
sharing systems already work at this scale order: bittorrent clients enable 
hundreds of thousands of users to share millions of file spread over the 
internet. If we disregard trackers, this protocol is totally decentralized with 
no single point of failure (SPOF). That is why we propose to learn from peer to 
peer file sharing experience, in order to build massively distributed clouds.

As at this scale failure becomes the norm rather the exception, it is vital to
take into account fault tolerance in the early stages of design, by leveraging a
peer to peer overlay network. As we think that working in a massively
distributed context require to deal with network parameters like 
latency and bandwidth usage, collaboration between the constituting nodes of the
system should be organized in a "network aware" manner. For instance, the 
Vivaldi algorithm \cite{dabek:2004:vivaldi} provides a dynamic coordinate system
that can be used to introduce locality properties inside a distributed system, 
thus building low latency collaborations.

We assume that the architecture of massively distributed clouds should be 
organized arround the same principles that rule communautary file sharing 
systems. That is why we propose in section \ref{sec:architecture} an 
architecture that will be build on top of peer to peer principles and 
articulated arround a locality based overlay network. To meet the high 
scalability criteria, some of IaaS mechanisms should be revisited to improve 
reactivity of inter-servers collaboration by leveraging locality properties.





\section{Revisiting OpenStack: towards a massively distributed IaaS manager}


\subsection{OpenStack: a toolkit for building clouds}

OpenStack is a very popular project that aim at building an opensource IaaS
manager. Many of the biggest actors of Cloud computing (Red hat, IBM, Rackspace,
VMware, Cisco, ...) are contributing to this project, thus providing new 
features at a rapide pace. OpenStack enables the construction of public or 
private clouds.

A typical cloud deployed with OpenStack is composed of several services (nova, 
swift, Quantum, Glance, ...) following the "shared nothing architecure" 
principle, meaning that each service is independent and thus shares no state 
with other services.

In this way, inter-services collaboration is performed by exchanging message 
through an AMQP (Advanced Messsage Queuing Protocol) based bus: each service 
has its own queue, and it collaborates with others by sending messages to their 
queues. This offers the advantage of easily plugging additional components : 
when a default service becomes no more suitable with system's needs, it can be 
naively replaced by another custom service, as long as the newer service 
consummes messages destinated to the older one, and in turn produces messages 
expected by collaborators.

As OpenStack is becoming a standard and since it provides mechansims that are
suitable to build clouds in data-centers context, it can be considerated as a
cloudkit of choice for the LUC-OS. 












\subsection{Revisiting components that are suitable for a LUC-OS (Related work)}
\label{sub:sec:revisiting_openstack}


As developing a complete IaaS manager is an Herculean work, we think that 
revisiting existing mechanims is a good way to reach our goal, even if it means 
that some of these mechanisms may be completely replaced.

For the sake of clarity, mechanisms that we plan to revisit will be studied 
following the list of services depicted in section \ref{sub:sec:list_services} :
for each service, the suitable mechanisms that will be revisited are indicated 
and proposed modification are detailled:

\begin{description}

	\item [Compute manager] : Nova is the OpenStack service that is 
	responsible for coordinating and controlling the work of other services.
	As it is the angular stone of OpenStack, we plan to directly plug the 
	LUC-OS on this service. Nova service contains a sub-service that is
	responsible for scheduling virtual machines: nova-scheduler, which performs
	static scheduling. This means that the scheduler will decide on which
	server the virtual machine will run once and for all. In section 
	\ref{sub:sec:integration_dvms} we discuss the replacement of 
	nova-scheduler by DVMS.


	\item [Administrative manager] : KeyStone and Horizon are two services
	that are respectively responsibles for managing identity/authentication
	and providing a web user interface to users that want to interact with
	OpenStack. As the LUC-OS will store every piece of information in a DHT: we
	propose to make KeyStone and Horizon pick data from this DHT. Furthermore 
	the LUC-OS and OpenStack does not exactly share the same semantics : for
	example the LUC-OS manipulates virtual environments which does not
	exactly exist in OpenStack. Revisiting Horizon service through minor
	modifications would enable the integration of this concept in OpenStack.

	\item [Storage manager] : Storage in OpenStack can be divided in two
	service: Swift manages objects (files) while Glance manages virtual machines
	disk images. Glance's images can be stored in Swift or in any other storage
	service. Recent solutions like VMTorrent \cite{reich:2012} have been
	developed to limit network overhead, for example during simultaneous 
	loading of images at VMs startup. For a first prototype we propose to 
	leverage Glance and Swift, and if it becomes no more suitable for a
	massive infrastucture, we propose to integrate VMTorrent in Glance and 
	to use it.

	\item [Network manager] : Neutron is the service that manages networking
	in OpenStack: it is used to create virtual networks (VLANs) that
	interconnect virtual machines. Neutron supports plugins : it is possible
	to use a virtual multilayer switch like Open vSwitch \cite{pfaff:2009}.
	In the case this plugin does not meet our needs, we propose to 
	leverage a software defined network (SDN) solution like Mininet.
	\cite{lantz:2010} or Vin.

\end{description}


%%%
\section{Conclusion\label{sec:conclusion}}

Distributing the management of Clouds is a solution to favor the adoption of the distributed cloud model. In this paper, we presented our view of how
such distribution can be achieved by presenting the premises of the LUC Operating System.
%We highlighted that it has  however a design cost
%and it  should be  developed over  mature and efficient  solutions:
We chose to develop it by leveraging the OpenStack solution. This choice
presents two advantages. It minimizes the development efforts and maximizes the
chance of being reused by a large community. As a proof-of-concept we presented
a revised version of the Nova service that uses a NoSQL backend. We discussed
few experiments validating the correct behavior and showed promising performance
over 8 clusters.
 
Our ongoing activities focus on two aspects. First, we expect
to finalize the same modifications on the Glance image service soon and start to
investigate also whether such a DB replacement can be achieved for Neutron. We
highlight that we chose to concentrate our effort on Glance as it is a key
element to operate an OpenStack IaaS platform. Indeed, while Neutron is becoming
more and more important, the historical network mechanisms integrated in Nova
are still available and intensively used. The second activity studies how it can
be possible to restrain the visibility of some objects manipulated by the
different controllers that have been deployed throughout the LUC infrastructure:
%Indeed some Finally, having a wan-wide infrastructure can be source of
%networking overheads:
our POC manipulates objects that might be used by any instance of a service, no
matter where it is deployed.
%On the  other hand, some  objects may
%benefit from a  restrained visibility:
If a user has build an OpenStack project (tenant) that is based on few sites,
appart from data-replication, there is no need for storing objects related to
this project on external sites. Restraining the storage of such objects
according to visibility rules would save network bandwidth and reduce
overheads.% in addition to enabling users to settle policies for applications
%such as privacy and efficient data-replication.

Although delivering an efficient distributed version of OpenStack is a
challenging task, we believe that addressing it is the key to go beyond
classical brokering/federated approaches and to promote a new generation of
cloud computing more sustainable and efficient. We are in touch with large
groups such as Orange Labs and are currently discussing with the OpenStack
foundation to propose the Rome/REDIS Library as an alternative to the
SQLAlchemy/MySQL couple. Most of the materials presented in this article such as
our prototype are available on the Discovery initiative website.

%Indeed,
%revising  OpenStack, in  order to  make  it natively  cooperative, would  enable
%Internet  Service Providers  and other  institutions  in charge  of operating  a
%network backbone  to build  an extreme-scale LUC  infrastructure with  a limited
%additional cost.

%% The  interest of important actors such as  Orange Labs that has
%% officially announced its  support to the initiative is an  excellent sign of the
%% importance of our action.       


%
% Section: Acknowledgment
%
\section*{Acknowledgments}

Since July 2015, the Discovery initiative is mainly supported through the Inria Project Labs program and the I/O labs, a joint lab between Inria and Orange Labs.
Further information at \href{http://www.inria.fr/en/research/research-fields/inria-project-labs}{http://www.inria.fr/en/research/research-fields/inria-project-labs}
%
% Bibliography
%



\bibliographystyle{abbrv}
\bibliography{main}
\end{document}
